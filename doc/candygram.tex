\documentclass{howto}
\title{Candygram}
\makeindex
\date{\today}
\author{Michael Hobbs}
\authoraddress{
        Email: \email{mike@hobbshouse.org}
}
\release{1.0 beta 1}
\setshortversion{1.0}

\newcommand{\Erlang}{\ulink{Erlang}{http://www.erlang.org/}}

% PDF output doesn't display '<' or '>' symbols well, unless it is in math mode.
% The math mode graphics generated by latex2html, on the other hand, look awful.
\ifpdf
 	\newcommand{\lessthan}[0]{\begin{math}<\end{math}}
 	\newcommand{\greaterthan}[0]{\begin{math}>\end{math}}
\else
 	\newcommand{\lessthan}[0]{<}
 	\newcommand{\greaterthan}[0]{>}
\fi



\begin{document}

\maketitle

Copyright \copyright\ 2004 Michael Hobbs.

Permission is granted to copy, distribute and/or modify this document
under the terms of the GNU Free Documentation License, Version 1.2
or any later version published by the Free Software Foundation;
with no Invariant Sections, no Front-Cover Texts, and no Back-Cover Texts.
A copy of the license is included in the section entitled ``GNU
Free Documentation License''. [\ref{fdl}]

\begin{abstract}
\noindent
Candygram is a Python implementation of \Erlang\ concurrency primitives. Erlang
is widely respected for its elegant built-in facilities for concurrent
programming. This package attempts to emulate those facilities as closely as
possible in Python. Specifically, this package provides functions that allow
developers to send and receive messages between threads using semantics nearly
identical to those in the Erlang language.
\end{abstract}

\tableofcontents



\section{Download}

\ulink{Download}{http://sourceforge.net/project/showfiles.php?group_id=114295}

Documentation is also available in PDF.



\section{Overview}

\begin{notice}[note]
Erlang uses a peculiar terminology in respect to threads. In Erlang parlance,
threads are actually called ``processes''. This terminology is due to a couple
of reasons. First, the Erlang runtime environment is a virtual machine. Second,
because Erlang is a functional language, no state is shared among its threads.
Since the concurrent tasks in an Erlang system run as peers within a single
[virtual] machine and don't share state, they are therefore named ``Processes''.

In order to avoid confusion when using this documentation alongside the Erlang
documentation, the word ``process'' is used instead of ``thread'' throughout the
remainder of this document.
\end{notice}

This package provides an implementation of the following \Erlang\ core
functions:
\begin{itemize}
	\item \function{spawn()}
	\item \function{send()}
	\item \keyword{receive}
	\item \function{link()}
\end{itemize}

These 4 functions form the core of Erlang's concurrency services. The
\function{spawn()} function creates a new process, the \function{send()}
function will send a message to another process, the \keyword{receive} statement
specifies what to do with received messages, and the \function{link()} function
allows one process to monitor the status of another process. In addition to
these core functions, this package also provides implementations of several
supplemental functions such as \function{spawn_link()} and \function{exit()}.

The beauty of the Erlang system is that it is simple and yet powerful.
To communicate with another process, you simply send a message to it.
You need not worry about locks, semaphores, mutexes, etc., in order to
share information among concurrent tasks. Even though message passing is
used typically to implement the producer/consumer model, when it is combined
with the flexibility of the \keyword{receive} statement, it becomes much more
powerful. For example, when using timeouts and selective receives, a process
may easily handle its messages as a state machine, or as a priority queue.

For those who wish to become more familiar with Erlang,
\citetitle[http://www.erlang.org/download/erlang-book-part1.pdf]
{Concurrent Programming in Erlang} provides a very complete introduction. In
particular, this package implements all of the functions described in chapter 5
and sections 7.2, 7.3, and 7.5 of that book.



\section{Receiver Patterns}

\newcommand{\addhandler}{\method{addHandler()}}

Erlang provides pattern matching in its very syntax, which is then used by the
\keyword{receive} statement. Since Python does not provide pattern matching in
its syntax, we must use a slightly different mechanism to match messages. The
first parameter passed to the \method{Receiver.addHandler()} method
[\ref{Receiver}] can be any Python value. The \addhandler\ method uses this
value as a pattern and interprets it in the following way:
\begin{enumerate}
	\item If the value is \constant{candygram.Any}, then any message will match.
	\item If the value is a class object or a type object, then any message that
		\function{isinstance()} of the class or type will match.
	\item If the value is a function or method, then a message will match if the
		function/method returns \constant{True} when called with the message.
	\item If the value is a list or tuple, then a message will match only if it
		is a list or tuple, respectively, and of the same length. Also, each value
		in the sequence is then used as a pattern and the sequence as a whole will
		match only if every pattern in the sequence matches.
	\item If the value is a list or tuple, and the last item in the sequence is
		\constant{candygram.AnyRemaining}, then the above rules apply except that
		the message sequence may be any length that is \greaterthan= len(pattern)-1.
	\item If the value is a dictionary, then a message will match only if it
		is a dictionary that contains all of the same keys as the pattern value.
		Also, each value in the dictionary is then used and a pattern and the
		dictionary as a whole will match only if every pattern in the dictionary
		matches its associated value in the message.
	\item Lastly, any other value is treated as a literal pattern. That is, a
		message will match if it is equal to the given value.
\end{enumerate}

\subsection{Examples}
These rules are best illustrated by example. In the table below, the first
column contains a Python value that is used as a pattern. The second column
contains Python values that would match the pattern and the third column
contains Python values that would not match the pattern.
\begin{tableiii}{l|l|l}{textrm}{Pattern}{Matches}{Non-Matches}
\lineiii{Any}
	{'shark', 13.7, (1, '', lambda: true)}
	{}
\lineiii{'shark'}
	{'shark'}
	{'dolphin', 43, []}
\lineiii{13.7}
	{13.7}
	{'shark', 13.6, \{'A': 14\}}
\lineiii{int}
	{13, 43, 0}
	{'shark', 13.7, []}
\lineiii{str}
	{'shark', '', 'dolphin'}
	{43, 0.9, lambda: true}
\lineiii{lambda x: x \greaterthan\ 20}
	{43, 100, 67.7}
	{13, 0, -67.7}
\lineiii{(str, int)}
	{('shark', 43), ('dolphin', 0)}
	{['shark', 43], ('dolphin', 43, 0.9)}
\lineiii{(str, int, AnyRemaining)}
	{('dolphin', 0), ('dolphin', 43, 0.9)}
	{('dolphin',), (43, 'dolphin')}
\lineiii{[str, 20, lambda x: x \lessthan\ 0]}
	{['shark', 20, -54.76], ['dolphin', 20, -1]}
	{['shark', 21, -6], [20, 20, -1], ['', 20]}
\lineiii{\{'S': int, 19: str\}}
	{\{'S': 3, 19: 'foo'\}, \{'S': -65, 19: 'bar', 'T': 'me'\}}
	{\{'S': 'Charlie', 19: 'foo'\}, \{'S': 3\}}
\end{tableiii}



\section{The \module{candygram} module}

\declaremodule{extension}{candygram}
\modulesynopsis{Erlang concurrency primitives}

The following functions, classes, and constants are exported by the
\module{candygram} module. Since the name \module{candygram} is a bit long, the
module is typically imported in one of the following ways:
\begin{verbatim}
>>> from candygram import *
\end{verbatim}
or
\begin{verbatim}
>>> import candygram as cg
\end{verbatim}


\subsection{Functions}

\begin{funcdesc}{spawn}{func\optional{, args\moreargs}}
Return the \class{Process} instance of a new concurrent process started by
calling the function \var{func} with the \var{args} argument list. When the
function returns, the process terminates. A \code{'badarg'}
\exception{ExitError} is raised if \var{func} is not \function{callable()}.
\end{funcdesc}

\begin{funcdesc}{link}{proc}
Create a link to the process \var{proc}, if there is not such a link already. If
a process attempts to create a link to itself, nothing is done. A
\code{'badarg'} \exception{ExitError} is raised if \var{proc} is not a
\class{Process} instance. A \code{'noproc'} \code{'EXIT'} signal is sent to
calling process if the \var{proc} process is no longer alive.

When a process terminates, an \code{'EXIT'} signal is sent to all of its linked
processes. If a process terminates normally, a \code{'normal'} \code{'EXIT'}
signal is sent to its linked processes.

All links are bidirectional. That is, if process A calls \code{link(B)}, then if
process B terminates, an \code{'EXIT'} signal is sent to process A. Conversely,
if process A terminates, an \code{'EXIT'} signal is likewise sent to process B.

Refer to the \function{processFlag()} function for details about handling
signals.
\end{funcdesc}

\begin{funcdesc}{spawnLink}{func\optional{, args\moreargs}}
This function is identical to the following code being evaluated in an atomic
operation:
\begin{verbatim}
>>> proc = spawn(func, args...)
>>> link(proc)
\end{verbatim}
This function is necessary since the process created might run immediately and
fail before \function{link()} is called. Returns the \class{Process} instance of
the newly created process. A \code{'badarg'} \exception{ExitError} is raised if
\var{func} is not \function{callable()}.
\end{funcdesc}

\begin{funcdesc}{unlink}{proc}
Remove the link, if there is one, from the calling process to another process
given by the \var{proc} argument. The function will not fail if the calling
process is not linked to \var{proc}, or if \var{proc} is not alive. A
\code{'badarg'} \exception{ExitError} is raised if \var{proc} is not a
\class{Process} instance.
\end{funcdesc}

\begin{funcdesc}{isProcessAlive}{proc}
Return \constant{True} if the process is alive, i.e., has not terminated.
Otherwise, return \constant{False}. A \code{'badarg'} \exception{ExitError} is
raised if \var{proc} is not a \class{Process} instance.
\end{funcdesc}

\begin{funcdesc}{self}{}
Return the \class{Process} instance of the calling process.
\end{funcdesc}

\begin{funcdesc}{self_}{}
An alias for the \function{self()} function. This function can be used used in
class methods where \var{self} is already defined.
\end{funcdesc}

\begin{funcdesc}{processes}{}
Return a list of all active processes.
\end{funcdesc}

\begin{funcdesc}{send}{proc, message}
Send the \var{message} to the \var{proc} process and return \var{message}. This
is the same as \var{proc}\code{.send(}\var{message}\code{)}. A \code{'badarg'}
\exception{ExitError} is raised if \var{proc} is not a \class{Process} instance.
\end{funcdesc}

\begin{funcdesc}{exit}{\optional{proc, }reason}
When the \var{proc} argument is not given, this function raises an
\exception{ExitError} with the reason \var{reason}. \var{reason} can be any
value.

When the \var{proc} argument is given, this function sends an \code{'EXIT'}
signal to the process \var{proc}. This is not necessarily the same as sending an
\code{'EXIT'} message to \var{proc}. They are the same if \var{proc} is trapping
exits. However, if \var{proc} is not trapping exits, the \var{proc} process will
exit and propagate the \code{'EXIT'} signal in turn to its linked processes.

If the \var{reason} is the string \code{'kill'}, for example
\code{exit(proc, 'kill')}, an untrappable \code{'EXIT'} signal will be sent to
the process. In other words, the \var{proc} process will be unconditionally
killed.

Refer to the \function{processFlag()} function for details about trapping exits.
\end{funcdesc}

\begin{funcdesc}{processFlag}{flag, option}
Set the given \var{flag} for the process which calls this function.
Returns the old value of the flag. A \code{'badarg'} \exception{ExitError} is
raised if \var{flag} is not a recognized flag value, or if \var{option} is not
a recognized value for \var{flag}.

Currently, the only recognized flag value is \code{'trap_exit'}. When
\code{'trap_exit'} is set to \constant{True}, \code{'EXIT'} signals arriving to
a process are converted to \code{('EXIT', from, reason)} messages, which can be
received as ordinary messages. If \code{'trap_exit'} is set to \constant{False},
the process exits if it receives an \code{'EXIT'} signal other than
\code{'normal'} and the \code{'EXIT'} signal is propagated to its linked
processes. Application processes should normally not trap exits.
\end{funcdesc}



\subsection{Process Objects}

\begin{classdesc}{Process}{}
Represents a concurrent process. A \class{Process} is never created via its
constructor. All processes are created via the \function{spawn()} and
\function{spawnLink()} functions instead.

\begin{methoddesc}{isProcessAlive}{}
Return \constant{True} if the process is alive, i.e., has not terminated.
Otherwise, return \constant{False}.
\end{methoddesc}

\begin{methoddesc}{isAlive}{}
An alias for the \method{isProcessAlive()} method. (The word ``process'' is
redundant in a method name in the \class{Process} class.)
\end{methoddesc}

\begin{methoddesc}{send}{message}
Send the \var{message} to this process and return \var{message}. Typically, a
routine running in a separate process calls this method to place the given
\var{message} into this process's mailbox. The message may then be picked up
by a \class{Receiver} that is operating in this process.

Sending a message is an asynchronous operation so the \method{send()} call will
not wait for the message to be received by a \class{Receiver}. Even if this
process has already terminated, the system will not notify the sender. Messages
are always delivered, and always in the same order they were sent.
\end{methoddesc}

\begin{methoddesc}{__or__}{message}
An alias for the \method{send()} method. The OR operator, `|', is aliased to
the \method{send()} method so that messages can be sent using a more Erlangy
syntax. In Erlang, messages are sent using the `!' primitive. For example:
\begin{verbatim}
>>> proc | ('knock-knock', 'candygram')
\end{verbatim}
\end{methoddesc}

\end{classdesc}



\subsection{Receiver Objects}
\label{Receiver}

\begin{classdesc}{Receiver}{}
% TODO: review documentation for send() and make sure this discription ties-in
% well with that documentation.
Some stuff about Receiver.

\begin{methoddesc}{addHandler}{pattern\optional{, func\optional{, \moreargs}}}

Some stuff about addHandler

When the \method{Receiver.receive()} method is called, the \class{Receiver}
object compares the first message in its mailbox with each pattern, in the order
that the handlers were added via the \addhandler\ method. If any pattern matches
the message, it will call the associated function and return its result.

If extra parameters are supplied in the call to \addhandler\ after the handler
function, those parameters are passed to the handler function when it is called.
If any of these function parameters is \constant{candygram.Message}, then the
parameter is replaced with the message that was received.
\end{methoddesc}

\end{classdesc}



\subsection{Constants}

\begin{datadesc}{Any}
\end{datadesc}

\begin{datadesc}{AnyRemaining}
\end{datadesc}

\begin{datadesc}{Message}
\end{datadesc}



\subsection{Exceptions}

\begin{excdesc}{ExitError}
\end{excdesc}



\section{Examples}

\begin{verbatim}
>>> from candygram import *
>>> r = Receiver()
>>> def open_door():
...     print 'Hello.'
...
>>> def shut_door():
...     print 'Go Away!'
...
>>> r.addHandler('shark', shut_door)
>>> r.addHandler('candygram', open_door)
>>> r.receive()
\end{verbatim}


\section{FAQ}

% \ifpdf
% 	\newcommand{\qna}[2]{
% 		\begin{tabular*}{\textwidth}{p{0.01\textwidth}p{0.95\textwidth}}
% 		\bfseries{}Q:&\bfseries{}#1\\
% 		A:&#2\\
% 		\end{tabular*}}
% \else
% 	\newcommand{\qna}[2]{
% 		\begin{tabular*}{\textwidth}{p{1em}l}
% 		\bfseries{}Q:&\bfseries{}#1\\
% 		A:&#2\\
% 		\end{tabular*}}
% \fi

\subsection{Why is the package called Candygram?}
The name Candygram is actually an acronym for ``the Candygram Acronym Does Not
Yield a Good Reference to Anything Meaningful.''

\subsection{But wait, doesn't that spell {\sc cadnygram}?}
Yes, you are quite observant. In order to form a compromise with the French
acronym, which is {\sc canydgram}, the official acronym was standardized as
{\sc Candygram}.

\subsection{How do you pronounce Candygram?}
This question produces an outrageous amount of heated debate. Some claim that
it is pronounced with short A's, as in tomato, while others claim that it is
pronounced with long A's, as in potato. Both sides, however, are completely
wrong; the name Candygram is correctly pronounced ``throat warbler mangrove.''



\section{Feedback}

Please submit bugs here



\appendix
\section{GNU Free Documentation License}
\label{fdl}
 \begin{center}

       Version 1.2, November 2002


 Copyright \copyright 2000,2001,2002  Free Software Foundation, Inc.

 \bigskip

     59 Temple Place, Suite 330, Boston, MA  02111-1307  USA

 \bigskip

 Everyone is permitted to copy and distribute verbatim copies
 of this license document, but changing it is not allowed.
\end{center}


\begin{center}
{\bf\large Preamble}
\end{center}

The purpose of this License is to make a manual, textbook, or other
functional and useful document "free" in the sense of freedom: to
assure everyone the effective freedom to copy and redistribute it,
with or without modifying it, either commercially or noncommercially.
Secondarily, this License preserves for the author and publisher a way
to get credit for their work, while not being considered responsible
for modifications made by others.

This License is a kind of "copyleft", which means that derivative
works of the document must themselves be free in the same sense.  It
complements the GNU General Public License, which is a copyleft
license designed for free software.

We have designed this License in order to use it for manuals for free
software, because free software needs free documentation: a free
program should come with manuals providing the same freedoms that the
software does.  But this License is not limited to software manuals;
it can be used for any textual work, regardless of subject matter or
whether it is published as a printed book.  We recommend this License
principally for works whose purpose is instruction or reference.


\begin{center}
{\Large\bf 1. APPLICABILITY AND DEFINITIONS}
\end{center}

This License applies to any manual or other work, in any medium, that
contains a notice placed by the copyright holder saying it can be
distributed under the terms of this License.  Such a notice grants a
world-wide, royalty-free license, unlimited in duration, to use that
work under the conditions stated herein.  The \textbf{"Document"}, below,
refers to any such manual or work.  Any member of the public is a
licensee, and is addressed as \textbf{"you"}.  You accept the license if you
copy, modify or distribute the work in a way requiring permission
under copyright law.

A \textbf{"Modified Version"} of the Document means any work containing the
Document or a portion of it, either copied verbatim, or with
modifications and/or translated into another language.

A \textbf{"Secondary Section"} is a named appendix or a front-matter section of
the Document that deals exclusively with the relationship of the
publishers or authors of the Document to the Document's overall subject
(or to related matters) and contains nothing that could fall directly
within that overall subject.  (Thus, if the Document is in part a
textbook of mathematics, a Secondary Section may not explain any
mathematics.)  The relationship could be a matter of historical
connection with the subject or with related matters, or of legal,
commercial, philosophical, ethical or political position regarding
them.

The \textbf{"Invariant Sections"} are certain Secondary Sections whose titles
are designated, as being those of Invariant Sections, in the notice
that says that the Document is released under this License.  If a
section does not fit the above definition of Secondary then it is not
allowed to be designated as Invariant.  The Document may contain zero
Invariant Sections.  If the Document does not identify any Invariant
Sections then there are none.

The \textbf{"Cover Texts"} are certain short passages of text that are listed,
as Front-Cover Texts or Back-Cover Texts, in the notice that says that
the Document is released under this License.  A Front-Cover Text may
be at most 5 words, and a Back-Cover Text may be at most 25 words.

A \textbf{"Transparent"} copy of the Document means a machine-readable copy,
represented in a format whose specification is available to the
general public, that is suitable for revising the document
straightforwardly with generic text editors or (for images composed of
pixels) generic paint programs or (for drawings) some widely available
drawing editor, and that is suitable for input to text formatters or
for automatic translation to a variety of formats suitable for input
to text formatters.  A copy made in an otherwise Transparent file
format whose markup, or absence of markup, has been arranged to thwart
or discourage subsequent modification by readers is not Transparent.
An image format is not Transparent if used for any substantial amount
of text.  A copy that is not "Transparent" is called \textbf{"Opaque"}.

Examples of suitable formats for Transparent copies include plain
ASCII without markup, Texinfo input format, LaTeX input format, SGML
or XML using a publicly available DTD, and standard-conforming simple
HTML, PostScript or PDF designed for human modification.  Examples of
transparent image formats include PNG, XCF and JPG.  Opaque formats
include proprietary formats that can be read and edited only by
proprietary word processors, SGML or XML for which the DTD and/or
processing tools are not generally available, and the
machine-generated HTML, PostScript or PDF produced by some word
processors for output purposes only.

The \textbf{"Title Page"} means, for a printed book, the title page itself,
plus such following pages as are needed to hold, legibly, the material
this License requires to appear in the title page.  For works in
formats which do not have any title page as such, "Title Page" means
the text near the most prominent appearance of the work's title,
preceding the beginning of the body of the text.

A section \textbf{"Entitled XYZ"} means a named subunit of the Document whose
title either is precisely XYZ or contains XYZ in parentheses following
text that translates XYZ in another language.  (Here XYZ stands for a
specific section name mentioned below, such as \textbf{"Acknowledgements"},
\textbf{"Dedications"}, \textbf{"Endorsements"}, or \textbf{"History"}.)
To \textbf{"Preserve the Title"}
of such a section when you modify the Document means that it remains a
section "Entitled XYZ" according to this definition.

The Document may include Warranty Disclaimers next to the notice which
states that this License applies to the Document.  These Warranty
Disclaimers are considered to be included by reference in this
License, but only as regards disclaiming warranties: any other
implication that these Warranty Disclaimers may have is void and has
no effect on the meaning of this License.


\begin{center}
{\Large\bf 2. VERBATIM COPYING}
\end{center}

You may copy and distribute the Document in any medium, either
commercially or noncommercially, provided that this License, the
copyright notices, and the license notice saying this License applies
to the Document are reproduced in all copies, and that you add no other
conditions whatsoever to those of this License.  You may not use
technical measures to obstruct or control the reading or further
copying of the copies you make or distribute.  However, you may accept
compensation in exchange for copies.  If you distribute a large enough
number of copies you must also follow the conditions in section 3.

You may also lend copies, under the same conditions stated above, and
you may publicly display copies.


\begin{center}
{\Large\bf 3. COPYING IN QUANTITY}
\end{center}


If you publish printed copies (or copies in media that commonly have
printed covers) of the Document, numbering more than 100, and the
Document's license notice requires Cover Texts, you must enclose the
copies in covers that carry, clearly and legibly, all these Cover
Texts: Front-Cover Texts on the front cover, and Back-Cover Texts on
the back cover.  Both covers must also clearly and legibly identify
you as the publisher of these copies.  The front cover must present
the full title with all words of the title equally prominent and
visible.  You may add other material on the covers in addition.
Copying with changes limited to the covers, as long as they preserve
the title of the Document and satisfy these conditions, can be treated
as verbatim copying in other respects.

If the required texts for either cover are too voluminous to fit
legibly, you should put the first ones listed (as many as fit
reasonably) on the actual cover, and continue the rest onto adjacent
pages.

If you publish or distribute Opaque copies of the Document numbering
more than 100, you must either include a machine-readable Transparent
copy along with each Opaque copy, or state in or with each Opaque copy
a computer-network location from which the general network-using
public has access to download using public-standard network protocols
a complete Transparent copy of the Document, free of added material.
If you use the latter option, you must take reasonably prudent steps,
when you begin distribution of Opaque copies in quantity, to ensure
that this Transparent copy will remain thus accessible at the stated
location until at least one year after the last time you distribute an
Opaque copy (directly or through your agents or retailers) of that
edition to the public.

It is requested, but not required, that you contact the authors of the
Document well before redistributing any large number of copies, to give
them a chance to provide you with an updated version of the Document.


\begin{center}
{\Large\bf 4. MODIFICATIONS}
\end{center}

You may copy and distribute a Modified Version of the Document under
the conditions of sections 2 and 3 above, provided that you release
the Modified Version under precisely this License, with the Modified
Version filling the role of the Document, thus licensing distribution
and modification of the Modified Version to whoever possesses a copy
of it.  In addition, you must do these things in the Modified Version:

\begin{itemize}
\item[A.]
   Use in the Title Page (and on the covers, if any) a title distinct
   from that of the Document, and from those of previous versions
   (which should, if there were any, be listed in the History section
   of the Document).  You may use the same title as a previous version
   if the original publisher of that version gives permission.

\item[B.]
   List on the Title Page, as authors, one or more persons or entities
   responsible for authorship of the modifications in the Modified
   Version, together with at least five of the principal authors of the
   Document (all of its principal authors, if it has fewer than five),
   unless they release you from this requirement.

\item[C.]
   State on the Title page the name of the publisher of the
   Modified Version, as the publisher.

\item[D.]
   Preserve all the copyright notices of the Document.

\item[E.]
   Add an appropriate copyright notice for your modifications
   adjacent to the other copyright notices.

\item[F.]
   Include, immediately after the copyright notices, a license notice
   giving the public permission to use the Modified Version under the
   terms of this License, in the form shown in the Addendum below.

\item[G.]
   Preserve in that license notice the full lists of Invariant Sections
   and required Cover Texts given in the Document's license notice.

\item[H.]
   Include an unaltered copy of this License.

\item[I.]
   Preserve the section Entitled "History", Preserve its Title, and add
   to it an item stating at least the title, year, new authors, and
   publisher of the Modified Version as given on the Title Page.  If
   there is no section Entitled "History" in the Document, create one
   stating the title, year, authors, and publisher of the Document as
   given on its Title Page, then add an item describing the Modified
   Version as stated in the previous sentence.

\item[J.]
   Preserve the network location, if any, given in the Document for
   public access to a Transparent copy of the Document, and likewise
   the network locations given in the Document for previous versions
   it was based on.  These may be placed in the "History" section.
   You may omit a network location for a work that was published at
   least four years before the Document itself, or if the original
   publisher of the version it refers to gives permission.

\item[K.]
   For any section Entitled "Acknowledgements" or "Dedications",
   Preserve the Title of the section, and preserve in the section all
   the substance and tone of each of the contributor acknowledgements
   and/or dedications given therein.

\item[L.]
   Preserve all the Invariant Sections of the Document,
   unaltered in their text and in their titles.  Section numbers
   or the equivalent are not considered part of the section titles.

\item[M.]
   Delete any section Entitled "Endorsements".  Such a section
   may not be included in the Modified Version.

\item[N.]
   Do not retitle any existing section to be Entitled "Endorsements"
   or to conflict in title with any Invariant Section.

\item[O.]
   Preserve any Warranty Disclaimers.
\end{itemize}

If the Modified Version includes new front-matter sections or
appendices that qualify as Secondary Sections and contain no material
copied from the Document, you may at your option designate some or all
of these sections as invariant.  To do this, add their titles to the
list of Invariant Sections in the Modified Version's license notice.
These titles must be distinct from any other section titles.

You may add a section Entitled "Endorsements", provided it contains
nothing but endorsements of your Modified Version by various
parties--for example, statements of peer review or that the text has
been approved by an organization as the authoritative definition of a
standard.

You may add a passage of up to five words as a Front-Cover Text, and a
passage of up to 25 words as a Back-Cover Text, to the end of the list
of Cover Texts in the Modified Version.  Only one passage of
Front-Cover Text and one of Back-Cover Text may be added by (or
through arrangements made by) any one entity.  If the Document already
includes a cover text for the same cover, previously added by you or
by arrangement made by the same entity you are acting on behalf of,
you may not add another; but you may replace the old one, on explicit
permission from the previous publisher that added the old one.

The author(s) and publisher(s) of the Document do not by this License
give permission to use their names for publicity for or to assert or
imply endorsement of any Modified Version.


\begin{center}
{\Large\bf 5. COMBINING DOCUMENTS}
\end{center}


You may combine the Document with other documents released under this
License, under the terms defined in section 4 above for modified
versions, provided that you include in the combination all of the
Invariant Sections of all of the original documents, unmodified, and
list them all as Invariant Sections of your combined work in its
license notice, and that you preserve all their Warranty Disclaimers.

The combined work need only contain one copy of this License, and
multiple identical Invariant Sections may be replaced with a single
copy.  If there are multiple Invariant Sections with the same name but
different contents, make the title of each such section unique by
adding at the end of it, in parentheses, the name of the original
author or publisher of that section if known, or else a unique number.
Make the same adjustment to the section titles in the list of
Invariant Sections in the license notice of the combined work.

In the combination, you must combine any sections Entitled "History"
in the various original documents, forming one section Entitled
"History"; likewise combine any sections Entitled "Acknowledgements",
and any sections Entitled "Dedications".  You must delete all sections
Entitled "Endorsements".

\begin{center}
{\Large\bf 6. COLLECTIONS OF DOCUMENTS}
\end{center}

You may make a collection consisting of the Document and other documents
released under this License, and replace the individual copies of this
License in the various documents with a single copy that is included in
the collection, provided that you follow the rules of this License for
verbatim copying of each of the documents in all other respects.

You may extract a single document from such a collection, and distribute
it individually under this License, provided you insert a copy of this
License into the extracted document, and follow this License in all
other respects regarding verbatim copying of that document.


\begin{center}
{\Large\bf 7. AGGREGATION WITH INDEPENDENT WORKS}
\end{center}


A compilation of the Document or its derivatives with other separate
and independent documents or works, in or on a volume of a storage or
distribution medium, is called an "aggregate" if the copyright
resulting from the compilation is not used to limit the legal rights
of the compilation's users beyond what the individual works permit.
When the Document is included in an aggregate, this License does not
apply to the other works in the aggregate which are not themselves
derivative works of the Document.

If the Cover Text requirement of section 3 is applicable to these
copies of the Document, then if the Document is less than one half of
the entire aggregate, the Document's Cover Texts may be placed on
covers that bracket the Document within the aggregate, or the
electronic equivalent of covers if the Document is in electronic form.
Otherwise they must appear on printed covers that bracket the whole
aggregate.


\begin{center}
{\Large\bf 8. TRANSLATION}
\end{center}


Translation is considered a kind of modification, so you may
distribute translations of the Document under the terms of section 4.
Replacing Invariant Sections with translations requires special
permission from their copyright holders, but you may include
translations of some or all Invariant Sections in addition to the
original versions of these Invariant Sections.  You may include a
translation of this License, and all the license notices in the
Document, and any Warranty Disclaimers, provided that you also include
the original English version of this License and the original versions
of those notices and disclaimers.  In case of a disagreement between
the translation and the original version of this License or a notice
or disclaimer, the original version will prevail.

If a section in the Document is Entitled "Acknowledgements",
"Dedications", or "History", the requirement (section 4) to Preserve
its Title (section 1) will typically require changing the actual
title.


\begin{center}
{\Large\bf 9. TERMINATION}
\end{center}


You may not copy, modify, sublicense, or distribute the Document except
as expressly provided for under this License.  Any other attempt to
copy, modify, sublicense or distribute the Document is void, and will
automatically terminate your rights under this License.  However,
parties who have received copies, or rights, from you under this
License will not have their licenses terminated so long as such
parties remain in full compliance.


\begin{center}
{\Large\bf 10. FUTURE REVISIONS OF THIS LICENSE}
\end{center}


The Free Software Foundation may publish new, revised versions
of the GNU Free Documentation License from time to time.  Such new
versions will be similar in spirit to the present version, but may
differ in detail to address new problems or concerns.  See
http://www.gnu.org/copyleft/.

Each version of the License is given a distinguishing version number.
If the Document specifies that a particular numbered version of this
License "or any later version" applies to it, you have the option of
following the terms and conditions either of that specified version or
of any later version that has been published (not as a draft) by the
Free Software Foundation.  If the Document does not specify a version
number of this License, you may choose any version ever published (not
as a draft) by the Free Software Foundation.


\begin{center}
{\Large\bf ADDENDUM: How to use this License for your documents}
\end{center}

To use this License in a document you have written, include a copy of
the License in the document and put the following copyright and
license notices just after the title page:

\bigskip
\begin{quote}
    Copyright \copyright  YEAR  YOUR NAME.
    Permission is granted to copy, distribute and/or modify this document
    under the terms of the GNU Free Documentation License, Version 1.2
    or any later version published by the Free Software Foundation;
    with no Invariant Sections, no Front-Cover Texts, and no Back-Cover Texts.
    A copy of the license is included in the section entitled "GNU
    Free Documentation License".
\end{quote}
\bigskip

If you have Invariant Sections, Front-Cover Texts and Back-Cover Texts,
replace the "with...Texts." line with this:

\bigskip
\begin{quote}
    with the Invariant Sections being LIST THEIR TITLES, with the
    Front-Cover Texts being LIST, and with the Back-Cover Texts being LIST.
\end{quote}
\bigskip

If you have Invariant Sections without Cover Texts, or some other
combination of the three, merge those two alternatives to suit the
situation.

If your document contains nontrivial examples of program code, we
recommend releasing these examples in parallel under your choice of
free software license, such as the GNU General Public License,
to permit their use in free software.




\documentclass{howto}
\title{Candygram}
\makeindex
\date{\today}
\author{Michael Hobbs}
\authoraddress{
        Email: \email{mike@hobbshouse.org}
}
\release{1.0 beta 1}
\setshortversion{1.0}

\newcommand{\Erlang}{\ulink{Erlang}{http://www.erlang.org/}}
\newcommand{\erlangbook}{\citetitle[http://www.erlang.org/download/erlang-book-part1.pdf]{Concurrent Programming in Erlang}}

% PDF output doesn't display some symbols well, unless it is in math mode.
% The math mode graphics generated by latex2html, on the other hand, look awful.
\ifpdf
 	\newcommand{\lessthan}[0]{\begin{math}<\end{math}}
 	\newcommand{\greaterthan}[0]{\begin{math}>\end{math}}
 	\newcommand{\pipe}[0]{\begin{math}|\end{math}}
\else
 	\newcommand{\lessthan}[0]{<}
 	\newcommand{\greaterthan}[0]{>}
 	\newcommand{\pipe}[0]{|}
\fi



\begin{document}

\maketitle

Copyright \copyright\ 2004 Michael Hobbs.

Permission is granted to copy, distribute and/or modify this document
under the terms of the GNU Free Documentation License, Version 1.2
or any later version published by the Free Software Foundation;
with no Invariant Sections, no Front-Cover Texts, and no Back-Cover Texts.
A copy of the license is included in the section entitled ``GNU
Free Documentation License''.

\begin{abstract}
\noindent
Candygram is a Python implementation of \Erlang\ concurrency primitives. Erlang
is widely respected for its elegant built-in facilities for concurrent
programming. This package attempts to emulate those facilities as closely as
possible in Python. With Candygram, developers can send and receive messages
between threads using semantics nearly identical to those in the Erlang
language.
\end{abstract}

\tableofcontents



% ############################################################################
\section{Download And Install}

You can download all available versions of Candygram from
\ulink{SourceForge.net}
	{http://sourceforge.net/project/showfiles.php?group_id=114295}. The current
stable release, as of this writing, is \version. You can also download this
document in PDF format from the link above.

To install, uncompress the zip or tar file, \program{cd} into the directory, and
run:
\begin{verbatim}
python setup.py install
\end{verbatim}

Windows users can just download the installer program and run it.



% ############################################################################
\section{Overview}

\begin{notice}[note]
Erlang uses a peculiar terminology in respect to threads. In Erlang parlance,
threads are called ``processes''. This terminology is due to a couple of
reasons. First, the Erlang runtime environment is a virtual machine. Second,
because Erlang is a functional language, no state is shared among its threads.
Since the concurrent tasks in an Erlang system run as peers within a single
[virtual] machine and don't share state, they are therefore named ``Processes''.

To avoid confusion when using this document alongside the Erlang documentation,
the remainder of this document uses the word ``process'' instead of ``thread''.
You are free, however, to pronounce the word ``process'' however you wish.
\end{notice}

This package provides an implementation of the following \Erlang\ core
functions:
\begin{itemize}
	\item \function{spawn()}
	\item \function{send()}
	\item \keyword{receive}
	\item \function{link()}
\end{itemize}

These 4 functions form the core of Erlang's concurrency services. The
\function{spawn()} function creates a new process, the \function{send()}
function sends a message to another process, the \keyword{receive} statement
specifies what to do with received messages, and the \function{link()} function
allows one process to monitor the status of another process. In addition to
these core functions, this package also provides implementations of several
supplemental functions such as \function{spawn_link()} and \function{exit()}.

The beauty of the Erlang system is that it is simple and yet powerful. To
communicate with another process, you simply send a message to it. You do not
need not worry about locks, semaphores, mutexes, etc. to share information among
concurrent tasks. Developers mostly use message passing only to implement the
producer/consumer model. When you combine message passing with the flexibility
of the \keyword{receive} statement, however, it becomes much more powerful. For
example, by using timeouts and receive patterns, a process may easily handle its
messages as a state machine, or as a priority queue.

For those who wish to become more familiar with Erlang, \erlangbook\ provides a
very complete introduction. In particular, this package implements all of the
functions described in chapter 5 and sections 7.2, 7.3, and 7.5 of that book.



% ############################################################################
\section{Receiver Patterns}
\label{patterns}
\index{patterns}

\newcommand{\addhandler}{\method{addHandler()}}

Erlang provides pattern matching in its very syntax, which the \keyword{receive}
statement uses to its advantage. Since Python does not provide pattern matching
in its syntax, we must use a slightly different mechanism to match messages. The
first parameter passed to the \method{Receiver.addHandler()} method
[\ref{Receiver}] can be any Python value. The \addhandler\ method uses this
value as a pattern and interprets it in the following way:
\begin{enumerate}
	\item If the value is \constant{candygram.Any}, then any message will match.
	\item If the value is a type object or a class, then any message that
		\function{isinstance()} of the type or class will match.
	\item If the value is \function{callable()}, then a message will match if the
		function/method returns \constant{True} when called with the message.
	\item If the value is a list or tuple, then a message will match only if it
		is a list or tuple, respectively, and of the same length. Also, each value
		in the sequence is used as a pattern and the sequence as a whole will
		match only if every pattern in the sequence matches its associated value in
		the message.
	\item If the value is a list or tuple, and the last item in the sequence is
		\constant{candygram.AnyRemaining}, then the above rule applies except that
		the message sequence may be any length that is \greaterthan= len(pattern)-1.
	\item If the value is a dictionary, then a message will match only if it
		is a dictionary that contains all of the same keys as the pattern value.
		Also, each value in the dictionary is used as a pattern and the
		dictionary as a whole will match only if every pattern in the dictionary
		matches its associated value in the message.
	\item Lastly, \addhandler\ treats any other value as a literal pattern. That
		is, a message will match if it is equal to the given value.
\end{enumerate}

\subsection{Examples}
Let's illustrate these rules by example. In the table below, the first column
contains a Python value that is used as a pattern. The second column contains
Python values that match the pattern and the third column contains Python values
that do not match the pattern.
\begin{tableiii}{l|l|l}{textrm}{Pattern}{Matches}{Non-Matches}
\lineiii{Any}
	{'shark', 13.7, (1, '', lambda: true)}
	{}
\lineiii{'shark'}
	{'shark'}
	{'dolphin', 42, []}
\lineiii{13.7}
	{13.7}
	{'shark', 13.6, \{'A': 14\}}
\lineiii{int}
	{13, 42, 0}
	{'shark', 13.7, []}
\lineiii{str}
	{'shark', '', 'dolphin'}
	{42, 0.9, lambda: True}
\lineiii{lambda x: x \greaterthan\ 20}
	{42, 100, 67.7}
	{13, 0, -67.7}
\lineiii{(str, int)}
	{('shark', 42), ('dolphin', 0)}
	{['shark', 42], ('dolphin', 42, 0)}
\lineiii{(str, int, AnyRemaining)}
	{('dolphin', 0), ('dolphin', 42, 0.9)}
	{('dolphin',), (42, 'dolphin')}
\lineiii{[str, 20, lambda x: x \lessthan\ 0]}
	{['shark', 20, -54.76], ['dolphin', 20, -1]}
	{['shark', 21, -6], [20, 20, -1], ['', 20]}
\lineiii{\{'S': int, 19: str\}}
	{\{'S': 3, 19: 'foo'\}, \{'S': -65, 19: 'bar', 'T': 'me'\}}
	{\{'S': 'Charlie', 19: 'foo'\}, \{'S': 3\}}
\end{tableiii}



% ############################################################################
\section{The \module{candygram} module}

\declaremodule{extension}{candygram}
\modulesynopsis{Erlang concurrency primitives}

The \module{candygram} module exports the following functions, classes,
constants, and exceptions. Since the name \module{candygram} is a bit long, you
would typically import the module in one of the following ways:
\begin{verbatim}
>>> from candygram import *
\end{verbatim}
or
\begin{verbatim}
>>> import candygram as cg
\end{verbatim}



% ----------------------------------------------------------------------------
\subsection{Functions}

\begin{funcdesc}{spawn}{func\optional{, args\moreargs}}
Create a new concurrent process by calling the function \var{func} with the
\var{args} argument list and return the resulting \class{Process} instance.
When the function returns, the process terminates. Raises a \code{'badarg'}
\exception{ExitError} if \var{func} is not \function{callable()}.
\end{funcdesc}

\begin{funcdesc}{link}{proc}
Create a link to the process \var{proc}, if there is not such a link already. If
a process attempts to create a link to itself, nothing is done. Raises a
\code{'badarg'} \exception{ExitError} if \var{proc} is not a \class{Process}
instance. Sends a \code{'noproc'} \code{'EXIT'} signal to calling process if the
\var{proc} process is no longer alive.

When a process terminates, it sends an \code{'EXIT'} signal is to all of its
linked processes. If a process terminates normally, it sends a \code{'normal'}
\code{'EXIT'} signal to its linked processes.

All links are bidirectional. That is, if process A calls \code{link(B)}, then if
process B terminates, it sends an \code{'EXIT'} signal to process A. Conversely,
if process A terminates, it likewise sends an \code{'EXIT'} signal to process B.

Refer to the \function{processFlag()} function for details about handling
signals.
\end{funcdesc}

\begin{funcdesc}{spawnLink}{func\optional{, args\moreargs}}
This function is identical to the following code being evaluated in an atomic
operation:
\begin{verbatim}
>>> proc = spawn(func, args...)
>>> link(proc)
\end{verbatim}
This function is necessary since the process created might run immediately and
fail before \function{link()} is called. Returns the \class{Process} instance of
the newly created process. Raises a \code{'badarg'} \exception{ExitError} if
\var{func} is not \function{callable()}.
\end{funcdesc}

\begin{funcdesc}{unlink}{proc}
Remove the link, if there is one, from the calling process to another process
given by the \var{proc} argument. The function does not fail if the calling
process is not linked to \var{proc}, or if \var{proc} is not alive. Raises a
\code{'badarg'} \exception{ExitError} if \var{proc} is not a \class{Process}
instance.
\end{funcdesc}

\begin{funcdesc}{isProcessAlive}{proc}
Return \constant{True} if the process is alive, i.e., has not terminated.
Otherwise, return \constant{False}. Raises a \code{'badarg'}
\exception{ExitError} if \var{proc} is not a \class{Process} instance.
\end{funcdesc}

\begin{funcdesc}{self}{}
Return the \class{Process} instance of the calling process.
\end{funcdesc}

\begin{funcdesc}{self_}{}
An alias for the \function{self()} function. You can use this function in class
methods where \var{self} is already defined.
\end{funcdesc}

\begin{funcdesc}{processes}{}
Return a list of all active processes.
\end{funcdesc}

\begin{funcdesc}{send}{proc, message}
Send the \var{message} to the \var{proc} process and return \var{message}. This
is the same as \var{proc}\code{.send(}\var{message}\code{)}. Raises a
\code{'badarg'} \exception{ExitError} if \var{proc} is not a \class{Process}
instance.
\end{funcdesc}

\begin{funcdesc}{exit}{\optional{proc, }reason}
When the \var{proc} argument is not given, this function raises an
\exception{ExitError} with the reason \var{reason}. \var{reason} can be any
value.

When the \var{proc} argument is given, this function sends an \code{'EXIT'}
signal to the process \var{proc}. This is not necessarily the same as sending an
\code{'EXIT'} message to \var{proc}. They are the same if \var{proc} is trapping
exits. If \var{proc} is not trapping exits, however, the \var{proc} process
terminates and propagates the \code{'EXIT'} signal in turn to its linked
processes.

If the \var{reason} is the string \code{'kill'}, for example
\code{exit(proc, 'kill')}, an untrappable \code{'EXIT'} signal is sent to the
process. In other words, the \var{proc} process is unconditionally killed.

Refer to the \function{processFlag()} function for details about trapping exits.
\end{funcdesc}

\begin{funcdesc}{processFlag}{flag, option}
Set the given \var{flag} for the calling process. Returns the old value of the
flag. Raises a \code{'badarg'} \exception{ExitError} if \var{flag} is not a
recognized flag value, or if \var{option} is not a recognized value for
\var{flag}.

Currently, \code{'trap_exit'} is the only recognized flag value. When
\code{'trap_exit'} is set to \constant{True}, \code{'EXIT'} signals arriving to
a process are converted to \code{('EXIT', from, reason)} messages, which can be
received as ordinary messages. If \code{'trap_exit'} is set to \constant{False},
the process exits if it receives an \code{'EXIT'} signal other than
\code{'normal'} and propagates the \code{'EXIT'} signal to its linked processes.
Application processes should normally not trap exits.
\end{funcdesc}



% ----------------------------------------------------------------------------
\subsection{Process Objects}

\begin{classdesc}{Process}{}
Represents a concurrent process. A \class{Process} is never created via its
constructor. The \function{spawn()} and \function{spawnLink()} functions create
all processes.

\begin{methoddesc}{isProcessAlive}{}
Return \constant{True} if the process is alive, i.e., has not terminated.
Otherwise, return \constant{False}.
\end{methoddesc}

\begin{methoddesc}{isAlive}{}
An alias for the \method{isProcessAlive()} method. (The word ``process'' is
redundant in a method name when the method is a member of the \class{Process}
class.)
\end{methoddesc}

\begin{methoddesc}{send}{message}
Send the \var{message} to this process and return \var{message}. A routine
running in a separate process calls this method to place the given
\var{message} into this process's mailbox. A \class{Receiver} that is operating
in this process may then pick up the message.

Sending a message is an asynchronous operation so the \method{send()} call does
not wait for a \class{Receiver} to retrieve the message. Even if this process
has already terminated, the system does not notify the sender. Messages are
always delivered, and always in the same order they were sent.
\end{methoddesc}

\begin{methoddesc}{__or__}{message}
\opindex{|}
An alias for the \method{send()} method. The OR operator, `\pipe', is aliased to
the \method{send()} method so that developers can use a more Erlangy syntax to
send messages. In Erlang, the `!' primitive sends messages. For example:
\begin{verbatim}
>>> proc | ('knock-knock', 'candygram')
\end{verbatim}
\end{methoddesc}

\end{classdesc}



% ----------------------------------------------------------------------------
\subsection{Receiver Objects}
\label{Receiver}

\begin{classdesc}{Receiver}{}
Retrieves messages from a process's mailbox. Every process maintains a mailbox,
which contains messages that have been sent to it via the \method{send()}
method. To retrieve the messages out of the mailbox, a process must use a
\class{Receiver} object. Like the \keyword{receive} statement in Erlang, a
\class{Receiver} object compares an incoming message against multiple patterns
[\ref{patterns}] and invokes the callback function that is associated
with the first matching pattern.

\begin{notice}[warning]
You may not use a single \class{Receiver} instance across multiple processes.
The \class{Receiver} raises an \exception{AssertionError} if you call one
of its methods from a process other than the one that created it. You should
take care, therefore, to create a \class{Receiver} within the process that will
be using it.
\end{notice}

\begin{methoddesc}{addHandler}{pattern\optional{, func\optional{, args\moreargs}}}
Register a handler function \var{func} for the \var{pattern}. The
\method{receive()} method calls the \var{func} with the \var{args} argument
list when it receives a message that matches \var{pattern}. When a handler
function \var{func} is not specified, the \method{receive()} method removes any
matching message from the mailbox and does nothing more with it. Refer to
section \ref{patterns} for details about how to specify message patterns. Raises
a \code{'badarg'} \exception{ExitError} if \var{func} is not
\function{callable()}.

If any of the \var{args} parameters is \constant{candygram.Message}, then the
\method{receive()} method replaces that parameter with the matching message when
it invokes \var{func}.

Refer to the \method{receive()} documentation for details about the
mechanism by which it invokes the handlers.
\end{methoddesc}

\begin{methoddesc}{__setitem__}{pattern, funcWithArgs}
\opindex{[]}
An alias for the \method{addHandler()} method. If \var{funcWithArgs} is a tuple,
then this method sends the first element in the tuple as the \var{func}
parameter to \method{addHandler()} and the remaining as the \var{args}. If
\var{funcWithArgs} is not a tuple, then this method assumes it to be a handler
function and sends it as the \var{func} parameter to \method{addHandler()}.

\method{__setitem__()} is an alias to the \method{addHandler()} method so that
developers can use a more Erlangy syntax to specify handlers. In Erlang, the
``\var{pattern} -\greaterthan\ \var{func}'' syntax specifies pattern guards. For
example:
\begin{verbatim}
>>> r = Receiver()
>>> r['knock-knock', 'candygram'] = answer, 'From whom?'
\end{verbatim}
\end{methoddesc}

\begin{methoddesc}{__getitem__}{pattern}
\opindex{[]}
An alias for the \method{addHandler()} method. This method calls
\method{addHandler()} without a \var{func} parameter.

\method{__getitem__()} is an alias to the \method{addHandler()} method so that
developers can use a more Erlangy syntax to specify handlers. In Erlang, the
``\var{pattern} -\greaterthan\ \var{func}'' syntax specifies pattern guards. For
example:
\begin{verbatim}
>>> r = Receiver()
>>> r['knock-knock', 'dolphin']  # ignore dolphins
\end{verbatim}
\end{methoddesc}

\begin{methoddesc}{addHandlers}{receiver}
Register all handler functions in \var{receiver} with this \class{Receiver}
object. This method adds all of the patterns and handler functions that have
been added to the given \var{receiver} to this receiver, in the same order. You
can use this method to make copies of \class{Receiver} objects. You may use a
\var{receiver} that was created in a different process. Raises a \code{'badarg'}
\exception{ExitError} if \var{receiver} is not a \class{Receiver} instance.
\end{methoddesc}

\begin{methoddesc}{after}{timeout\optional{, func\optional{, args\moreargs}}}
Register a timeout handler function \var{func}. The \method{receive()} method
calls the \var{func} with the \var{args} argument list when it does not receive
a matching message within \var{timeout} milliseconds. Raises a \code{'badarg'}
\exception{ExitError} if \var{func} is not \function{callable()} or if
\var{timeout} is not an integer. Raises an \exception{AssertionError} if the
\method{after()} method has already been invoked.

If the \method{receive()} method does not find a matching message within
\var{timeout} milliseconds, it invokes \var{func} with \var{args} and returns
the result. When a handler function \var{func} is not specified, the
\method{receive()} method returns \constant{None} if it times out.

None of the \var{args} parameters should be \constant{candygram.Message}, since
there is no message to pass when a timeout occurs.
\end{methoddesc}

\begin{methoddesc}{receive}{\optional{timeout\optional{, func\optional{, args\moreargs}}}}
Find a matching message in the process's mailbox, invoke the related handler
function, and return the result. The \class{Receiver} object compares the first
message in the process's mailbox with each of its registered patterns, in the
order that the patterns were added via the \method{addHandler()} method. If any
pattern matches the message, this method removes the message from the mailbox,
calls the associated handler function, and returns its result. If the matching
pattern does not have a handler function associated with it, this method returns
\constant{None}. If no pattern matches the message, this method will leave the
message in the mailbox, skips to the next message in the mailbox, and compares
it with each of the patterns. It continues on through each of the messages in
the mailbox until a match is found.

If no message matches any of the patterns, or if the mailbox is empty, this
method blocks until the process receives a message that does match one of the
patterns, or until the given \var{timeout} has elapsed. If no \var{timeout} is
specified, this method blocks indefinitely. If a \var{timeout} is specified, it
is equivalent to calling \method{after(\var{timeout}, \var{func}, \var{args})}
just prior to \method{receive()}. This method raises a \code{'badarg'}
\exception{ExitError}, therefore, if \var{func} is not \function{callable()} or
if \var{timeout} is not an integer. It also raises an \exception{AssertionError}
if a \var{timeout} is specified and the \method{after()} method has already been
invoked.
\end{methoddesc}

\begin{methoddesc}{__call__}{\optional{timeout\optional{, func\optional{, args\moreargs}}}}
\opindex{()}
An alias for the \method{receive()} method. \method{__call__()} is an alias to
the \method{receive()} method so that developers can use a shortened, more
Erlangy syntax with \class{Receiver}s. For example:
\begin{verbatim}
>>> def convert():
...     r = Receiver()
...     r['one'] = lambda: 1
...     r['two'] = lambda: 2
...     r['three'] = lambda: 3
...     return r()
\end{verbatim}
\end{methoddesc}

\begin{methoddesc}{__iter__}{}
Return an iterator that repeatedly invokes \method{receive()}. Candygram code
often uses the following idiom:
\begin{verbatim}
... while True:
...     result = receiver.receive()
...     # do whatever with the result...
\end{verbatim}
With an iterator, you can spell the code above like this instead:
\begin{verbatim}
... for result in receiver:
...     # do whatever with the result...
\end{verbatim}
\end{methoddesc}

\end{classdesc}



% ----------------------------------------------------------------------------
\subsection{Constants}

\begin{datadesc}{Any}
When used in a pattern, the \constant{Any} constant will match any value.
\end{datadesc}

\begin{datadesc}{AnyRemaining}
When used in the last position of a tuple or list pattern, the
\constant{AnyRemaining} constant will cause the sequence to match a sequence of
the same kind (tuple or list) that has a length \greaterthan= len(pattern)-1.
The patterns in the sequence prior to \constant{AnyRemaining} must still match
their respective values in a message for the whole sequence to match.
\end{datadesc}

\begin{datadesc}{Message}
When used as a function parameter in \method{Receiver.addHandler()}, the
\method{Receiver.receive()} method replaces the parameter with the matching
message when it invokes the handler function.
\end{datadesc}



% ----------------------------------------------------------------------------
\subsection{Exceptions}

\begin{excdesc}{ExitError}
Represents \code{'EXIT'} errors from Erlang. If an Erlang function can
cause a failure under certain circumstances, then the corresponding Candygram
function raises an \exception{ExitError} under the same circumstances.

A process also raises an \exception{ExitError} in all of its linked processes
if it terminates for a reason other than \code{'normal'}.

\begin{notice}[note]
When a process terminates with a reason other than \code{'normal'}, it does not
immediately raise an \exception{ExitError} in all linked processes. Candygram
defers the \exception{ExitError} instead, and raises it the next time one of its
functions or methods is called. (Python does not allow you to unconditionally
interrupt a separate thread.)
\end{notice}

\begin{memberdesc}{reason}
Specifies the reason why the \exception{ExitError} was raised. Can be any value.
\end{memberdesc}

\begin{memberdesc}{proc}
The process that originally caused the \exception{ExitError}.
\end{memberdesc}

\end{excdesc}



% ############################################################################
\section{Examples}
There is a directory named \file{examples} within every distribution of
Candygram. This directory contains the Candygram equivalents of all the sample
programs found in chapter 5 and sections 7.2, 7.3, and 7.5 of \erlangbook. The
files named \file{program_X.X.py} are direct translations of the Erlang
programs. The files named \file{program_X.X_alt.py} are alternate, more liberal
translations that use more Pythonic idioms. \note{Since the CPython interpreter
does not perform tail-call optimization, the tail-recursive style of the direct
translations is not a recommended practice.}

The \file{examples} directory also contains a few other modules that demonstrate
how you can Candygram to perform some handy functions.

If you are not already familiar with Erlang, the best way to become familiar
with Candygram is to read chapter 5 in \erlangbook\ and follow along using the
example programs located in the \file{examples} directory. If you are already
familiar with Erlang, here are a few code snippets to give you a taste of
Candygram.

\begin{verbatim}
>>> from candygram import *
>>> def proc_func():
...     r = Receiver()
...     r['land shark'] = lambda m: 'Go Away ' + m, Message
...     r['candygram'] = lambda m: 'Hello ' + m, Message
...     for message in r:
...         print message
...
>>> proc = spawn(proc_func)
>>> proc | 'land shark'
>>> proc | 'candygram'
\end{verbatim}
Running the code above produces the following output:
\begin{verbatim}
Go Away land shark
Hello candygram
\end{verbatim}

The code above uses a rather Erlangy syntax. Here is a more Pythonic version
that does the same:
\begin{verbatim}
>>> import candygram as cg
>>> def proc_func():
...     r = cg.Receiver()
...     r.addHandler('land shark', shut_door, cg.Message)
...     r.addHandler('candygram', open_door, cg.Message)
...     for message in r:
...         print message
...
>>> def shut_door(name):
...     return 'Go Away ' + name
...
>>> def open_door(name):
...     return 'Hello ' + name
...
>>> proc = cg.spawn(proc_func)
>>> proc.send('land shark')
>>> proc.send('candygram')
\end{verbatim}

Lastly, here is an example with more elaborate patterns:
\begin{verbatim}
>>> from candygram import *
>>> def proc_func(name):
...     r = Receiver()
...     r['msg', Process, str] = print_string, name, Message
...     r['msg', Process, str, Any] = print_any, name, Message
...     r[Any]  # Ignore any other messages
...     for result in r:
...         pass
...
>>> def print_string(name, message):
...     msg, process, string = message
...     # 'msg' and 'process' are unused
...     print '%s received: %s' % (name, string)
...
>>> def print_any(name, message):
...     msg, process, prefix, value = message
...     # 'msg' and 'process' are unused
...     print '%s received: %s %s' % (name, prefix, value)
...
>>> a = spawn(proc_func, 'A')
>>> b = spawn(proc_func, 'B')
>>> a | ('msg', b, 'Girl Scout cookies')
>>> a | 'plumber?'
>>> a | ('msg', b, 'The meaning of life is:', 42)
\end{verbatim}
Running the code above produces the following output:
\begin{verbatim}
A received: Girl Scout cookies
A received: The meaning of life is: 42
\end{verbatim}



% ############################################################################
\section{FAQ}

\subsection{Why is the package called Candygram?}
The name Candygram is actually an acronym for ``the Candygram Acronym Does Not
Yield a Good Reference to Anything Meaningful.''

\subsection{But wait, doesn't that spell {\sc cadnygram}?}
Yes, you are quite observant. In order to form a compromise with the French
acronym, which is {\sc canydgram}, the Candygram committee standardized the
official acronym as {\sc Candygram}.

\subsection{How do you pronounce Candygram?}
This question produces an outrageous amount of heated debate. Some claim that
you pronounce it with short A's, as in tomato, while others claim that you
pronounce it with long A's, as in potato. Both sides, however, are completely
wrong; the correct pronunciation for Candygram is ``throat warbler mangrove.''



% ############################################################################
\section{Feedback}

Please submit all bug reports, feature requests, etc. to the appropriate tracker
on \ulink{SourceForge.net}{http://sourceforge.net/tracker/?group_id=114295}.

General discussion takes place on the
\email{candygram-discuss@lists.sourceforge.net} mailing list. You can subscribe
to this list by visiting \ulink{this page}
	{http://lists.sourceforge.net/lists/listinfo/candygram-discuss}.



% ############################################################################
\appendix
\section{GNU Free Documentation License}
\label{fdl}
 \begin{center}

       Version 1.2, November 2002


 Copyright \copyright 2000,2001,2002  Free Software Foundation, Inc.

 \bigskip

     59 Temple Place, Suite 330, Boston, MA  02111-1307  USA

 \bigskip

 Everyone is permitted to copy and distribute verbatim copies
 of this license document, but changing it is not allowed.
\end{center}


\begin{center}
{\bf\large Preamble}
\end{center}

The purpose of this License is to make a manual, textbook, or other
functional and useful document "free" in the sense of freedom: to
assure everyone the effective freedom to copy and redistribute it,
with or without modifying it, either commercially or noncommercially.
Secondarily, this License preserves for the author and publisher a way
to get credit for their work, while not being considered responsible
for modifications made by others.

This License is a kind of "copyleft", which means that derivative
works of the document must themselves be free in the same sense.  It
complements the GNU General Public License, which is a copyleft
license designed for free software.

We have designed this License in order to use it for manuals for free
software, because free software needs free documentation: a free
program should come with manuals providing the same freedoms that the
software does.  But this License is not limited to software manuals;
it can be used for any textual work, regardless of subject matter or
whether it is published as a printed book.  We recommend this License
principally for works whose purpose is instruction or reference.


\begin{center}
{\Large\bf 1. APPLICABILITY AND DEFINITIONS}
\end{center}

This License applies to any manual or other work, in any medium, that
contains a notice placed by the copyright holder saying it can be
distributed under the terms of this License.  Such a notice grants a
world-wide, royalty-free license, unlimited in duration, to use that
work under the conditions stated herein.  The \textbf{"Document"}, below,
refers to any such manual or work.  Any member of the public is a
licensee, and is addressed as \textbf{"you"}.  You accept the license if you
copy, modify or distribute the work in a way requiring permission
under copyright law.

A \textbf{"Modified Version"} of the Document means any work containing the
Document or a portion of it, either copied verbatim, or with
modifications and/or translated into another language.

A \textbf{"Secondary Section"} is a named appendix or a front-matter section of
the Document that deals exclusively with the relationship of the
publishers or authors of the Document to the Document's overall subject
(or to related matters) and contains nothing that could fall directly
within that overall subject.  (Thus, if the Document is in part a
textbook of mathematics, a Secondary Section may not explain any
mathematics.)  The relationship could be a matter of historical
connection with the subject or with related matters, or of legal,
commercial, philosophical, ethical or political position regarding
them.

The \textbf{"Invariant Sections"} are certain Secondary Sections whose titles
are designated, as being those of Invariant Sections, in the notice
that says that the Document is released under this License.  If a
section does not fit the above definition of Secondary then it is not
allowed to be designated as Invariant.  The Document may contain zero
Invariant Sections.  If the Document does not identify any Invariant
Sections then there are none.

The \textbf{"Cover Texts"} are certain short passages of text that are listed,
as Front-Cover Texts or Back-Cover Texts, in the notice that says that
the Document is released under this License.  A Front-Cover Text may
be at most 5 words, and a Back-Cover Text may be at most 25 words.

A \textbf{"Transparent"} copy of the Document means a machine-readable copy,
represented in a format whose specification is available to the
general public, that is suitable for revising the document
straightforwardly with generic text editors or (for images composed of
pixels) generic paint programs or (for drawings) some widely available
drawing editor, and that is suitable for input to text formatters or
for automatic translation to a variety of formats suitable for input
to text formatters.  A copy made in an otherwise Transparent file
format whose markup, or absence of markup, has been arranged to thwart
or discourage subsequent modification by readers is not Transparent.
An image format is not Transparent if used for any substantial amount
of text.  A copy that is not "Transparent" is called \textbf{"Opaque"}.

Examples of suitable formats for Transparent copies include plain
ASCII without markup, Texinfo input format, LaTeX input format, SGML
or XML using a publicly available DTD, and standard-conforming simple
HTML, PostScript or PDF designed for human modification.  Examples of
transparent image formats include PNG, XCF and JPG.  Opaque formats
include proprietary formats that can be read and edited only by
proprietary word processors, SGML or XML for which the DTD and/or
processing tools are not generally available, and the
machine-generated HTML, PostScript or PDF produced by some word
processors for output purposes only.

The \textbf{"Title Page"} means, for a printed book, the title page itself,
plus such following pages as are needed to hold, legibly, the material
this License requires to appear in the title page.  For works in
formats which do not have any title page as such, "Title Page" means
the text near the most prominent appearance of the work's title,
preceding the beginning of the body of the text.

A section \textbf{"Entitled XYZ"} means a named subunit of the Document whose
title either is precisely XYZ or contains XYZ in parentheses following
text that translates XYZ in another language.  (Here XYZ stands for a
specific section name mentioned below, such as \textbf{"Acknowledgements"},
\textbf{"Dedications"}, \textbf{"Endorsements"}, or \textbf{"History"}.)
To \textbf{"Preserve the Title"}
of such a section when you modify the Document means that it remains a
section "Entitled XYZ" according to this definition.

The Document may include Warranty Disclaimers next to the notice which
states that this License applies to the Document.  These Warranty
Disclaimers are considered to be included by reference in this
License, but only as regards disclaiming warranties: any other
implication that these Warranty Disclaimers may have is void and has
no effect on the meaning of this License.


\begin{center}
{\Large\bf 2. VERBATIM COPYING}
\end{center}

You may copy and distribute the Document in any medium, either
commercially or noncommercially, provided that this License, the
copyright notices, and the license notice saying this License applies
to the Document are reproduced in all copies, and that you add no other
conditions whatsoever to those of this License.  You may not use
technical measures to obstruct or control the reading or further
copying of the copies you make or distribute.  However, you may accept
compensation in exchange for copies.  If you distribute a large enough
number of copies you must also follow the conditions in section 3.

You may also lend copies, under the same conditions stated above, and
you may publicly display copies.


\begin{center}
{\Large\bf 3. COPYING IN QUANTITY}
\end{center}


If you publish printed copies (or copies in media that commonly have
printed covers) of the Document, numbering more than 100, and the
Document's license notice requires Cover Texts, you must enclose the
copies in covers that carry, clearly and legibly, all these Cover
Texts: Front-Cover Texts on the front cover, and Back-Cover Texts on
the back cover.  Both covers must also clearly and legibly identify
you as the publisher of these copies.  The front cover must present
the full title with all words of the title equally prominent and
visible.  You may add other material on the covers in addition.
Copying with changes limited to the covers, as long as they preserve
the title of the Document and satisfy these conditions, can be treated
as verbatim copying in other respects.

If the required texts for either cover are too voluminous to fit
legibly, you should put the first ones listed (as many as fit
reasonably) on the actual cover, and continue the rest onto adjacent
pages.

If you publish or distribute Opaque copies of the Document numbering
more than 100, you must either include a machine-readable Transparent
copy along with each Opaque copy, or state in or with each Opaque copy
a computer-network location from which the general network-using
public has access to download using public-standard network protocols
a complete Transparent copy of the Document, free of added material.
If you use the latter option, you must take reasonably prudent steps,
when you begin distribution of Opaque copies in quantity, to ensure
that this Transparent copy will remain thus accessible at the stated
location until at least one year after the last time you distribute an
Opaque copy (directly or through your agents or retailers) of that
edition to the public.

It is requested, but not required, that you contact the authors of the
Document well before redistributing any large number of copies, to give
them a chance to provide you with an updated version of the Document.


\begin{center}
{\Large\bf 4. MODIFICATIONS}
\end{center}

You may copy and distribute a Modified Version of the Document under
the conditions of sections 2 and 3 above, provided that you release
the Modified Version under precisely this License, with the Modified
Version filling the role of the Document, thus licensing distribution
and modification of the Modified Version to whoever possesses a copy
of it.  In addition, you must do these things in the Modified Version:

\begin{itemize}
\item[A.]
   Use in the Title Page (and on the covers, if any) a title distinct
   from that of the Document, and from those of previous versions
   (which should, if there were any, be listed in the History section
   of the Document).  You may use the same title as a previous version
   if the original publisher of that version gives permission.

\item[B.]
   List on the Title Page, as authors, one or more persons or entities
   responsible for authorship of the modifications in the Modified
   Version, together with at least five of the principal authors of the
   Document (all of its principal authors, if it has fewer than five),
   unless they release you from this requirement.

\item[C.]
   State on the Title page the name of the publisher of the
   Modified Version, as the publisher.

\item[D.]
   Preserve all the copyright notices of the Document.

\item[E.]
   Add an appropriate copyright notice for your modifications
   adjacent to the other copyright notices.

\item[F.]
   Include, immediately after the copyright notices, a license notice
   giving the public permission to use the Modified Version under the
   terms of this License, in the form shown in the Addendum below.

\item[G.]
   Preserve in that license notice the full lists of Invariant Sections
   and required Cover Texts given in the Document's license notice.

\item[H.]
   Include an unaltered copy of this License.

\item[I.]
   Preserve the section Entitled "History", Preserve its Title, and add
   to it an item stating at least the title, year, new authors, and
   publisher of the Modified Version as given on the Title Page.  If
   there is no section Entitled "History" in the Document, create one
   stating the title, year, authors, and publisher of the Document as
   given on its Title Page, then add an item describing the Modified
   Version as stated in the previous sentence.

\item[J.]
   Preserve the network location, if any, given in the Document for
   public access to a Transparent copy of the Document, and likewise
   the network locations given in the Document for previous versions
   it was based on.  These may be placed in the "History" section.
   You may omit a network location for a work that was published at
   least four years before the Document itself, or if the original
   publisher of the version it refers to gives permission.

\item[K.]
   For any section Entitled "Acknowledgements" or "Dedications",
   Preserve the Title of the section, and preserve in the section all
   the substance and tone of each of the contributor acknowledgements
   and/or dedications given therein.

\item[L.]
   Preserve all the Invariant Sections of the Document,
   unaltered in their text and in their titles.  Section numbers
   or the equivalent are not considered part of the section titles.

\item[M.]
   Delete any section Entitled "Endorsements".  Such a section
   may not be included in the Modified Version.

\item[N.]
   Do not retitle any existing section to be Entitled "Endorsements"
   or to conflict in title with any Invariant Section.

\item[O.]
   Preserve any Warranty Disclaimers.
\end{itemize}

If the Modified Version includes new front-matter sections or
appendices that qualify as Secondary Sections and contain no material
copied from the Document, you may at your option designate some or all
of these sections as invariant.  To do this, add their titles to the
list of Invariant Sections in the Modified Version's license notice.
These titles must be distinct from any other section titles.

You may add a section Entitled "Endorsements", provided it contains
nothing but endorsements of your Modified Version by various
parties--for example, statements of peer review or that the text has
been approved by an organization as the authoritative definition of a
standard.

You may add a passage of up to five words as a Front-Cover Text, and a
passage of up to 25 words as a Back-Cover Text, to the end of the list
of Cover Texts in the Modified Version.  Only one passage of
Front-Cover Text and one of Back-Cover Text may be added by (or
through arrangements made by) any one entity.  If the Document already
includes a cover text for the same cover, previously added by you or
by arrangement made by the same entity you are acting on behalf of,
you may not add another; but you may replace the old one, on explicit
permission from the previous publisher that added the old one.

The author(s) and publisher(s) of the Document do not by this License
give permission to use their names for publicity for or to assert or
imply endorsement of any Modified Version.


\begin{center}
{\Large\bf 5. COMBINING DOCUMENTS}
\end{center}


You may combine the Document with other documents released under this
License, under the terms defined in section 4 above for modified
versions, provided that you include in the combination all of the
Invariant Sections of all of the original documents, unmodified, and
list them all as Invariant Sections of your combined work in its
license notice, and that you preserve all their Warranty Disclaimers.

The combined work need only contain one copy of this License, and
multiple identical Invariant Sections may be replaced with a single
copy.  If there are multiple Invariant Sections with the same name but
different contents, make the title of each such section unique by
adding at the end of it, in parentheses, the name of the original
author or publisher of that section if known, or else a unique number.
Make the same adjustment to the section titles in the list of
Invariant Sections in the license notice of the combined work.

In the combination, you must combine any sections Entitled "History"
in the various original documents, forming one section Entitled
"History"; likewise combine any sections Entitled "Acknowledgements",
and any sections Entitled "Dedications".  You must delete all sections
Entitled "Endorsements".

\begin{center}
{\Large\bf 6. COLLECTIONS OF DOCUMENTS}
\end{center}

You may make a collection consisting of the Document and other documents
released under this License, and replace the individual copies of this
License in the various documents with a single copy that is included in
the collection, provided that you follow the rules of this License for
verbatim copying of each of the documents in all other respects.

You may extract a single document from such a collection, and distribute
it individually under this License, provided you insert a copy of this
License into the extracted document, and follow this License in all
other respects regarding verbatim copying of that document.


\begin{center}
{\Large\bf 7. AGGREGATION WITH INDEPENDENT WORKS}
\end{center}


A compilation of the Document or its derivatives with other separate
and independent documents or works, in or on a volume of a storage or
distribution medium, is called an "aggregate" if the copyright
resulting from the compilation is not used to limit the legal rights
of the compilation's users beyond what the individual works permit.
When the Document is included in an aggregate, this License does not
apply to the other works in the aggregate which are not themselves
derivative works of the Document.

If the Cover Text requirement of section 3 is applicable to these
copies of the Document, then if the Document is less than one half of
the entire aggregate, the Document's Cover Texts may be placed on
covers that bracket the Document within the aggregate, or the
electronic equivalent of covers if the Document is in electronic form.
Otherwise they must appear on printed covers that bracket the whole
aggregate.


\begin{center}
{\Large\bf 8. TRANSLATION}
\end{center}


Translation is considered a kind of modification, so you may
distribute translations of the Document under the terms of section 4.
Replacing Invariant Sections with translations requires special
permission from their copyright holders, but you may include
translations of some or all Invariant Sections in addition to the
original versions of these Invariant Sections.  You may include a
translation of this License, and all the license notices in the
Document, and any Warranty Disclaimers, provided that you also include
the original English version of this License and the original versions
of those notices and disclaimers.  In case of a disagreement between
the translation and the original version of this License or a notice
or disclaimer, the original version will prevail.

If a section in the Document is Entitled "Acknowledgements",
"Dedications", or "History", the requirement (section 4) to Preserve
its Title (section 1) will typically require changing the actual
title.


\begin{center}
{\Large\bf 9. TERMINATION}
\end{center}


You may not copy, modify, sublicense, or distribute the Document except
as expressly provided for under this License.  Any other attempt to
copy, modify, sublicense or distribute the Document is void, and will
automatically terminate your rights under this License.  However,
parties who have received copies, or rights, from you under this
License will not have their licenses terminated so long as such
parties remain in full compliance.


\begin{center}
{\Large\bf 10. FUTURE REVISIONS OF THIS LICENSE}
\end{center}


The Free Software Foundation may publish new, revised versions
of the GNU Free Documentation License from time to time.  Such new
versions will be similar in spirit to the present version, but may
differ in detail to address new problems or concerns.  See
http://www.gnu.org/copyleft/.

Each version of the License is given a distinguishing version number.
If the Document specifies that a particular numbered version of this
License "or any later version" applies to it, you have the option of
following the terms and conditions either of that specified version or
of any later version that has been published (not as a draft) by the
Free Software Foundation.  If the Document does not specify a version
number of this License, you may choose any version ever published (not
as a draft) by the Free Software Foundation.


\begin{center}
{\Large\bf ADDENDUM: How to use this License for your documents}
\end{center}

To use this License in a document you have written, include a copy of
the License in the document and put the following copyright and
license notices just after the title page:

\bigskip
\begin{quote}
    Copyright \copyright  YEAR  YOUR NAME.
    Permission is granted to copy, distribute and/or modify this document
    under the terms of the GNU Free Documentation License, Version 1.2
    or any later version published by the Free Software Foundation;
    with no Invariant Sections, no Front-Cover Texts, and no Back-Cover Texts.
    A copy of the license is included in the section entitled "GNU
    Free Documentation License".
\end{quote}
\bigskip

If you have Invariant Sections, Front-Cover Texts and Back-Cover Texts,
replace the "with...Texts." line with this:

\bigskip
\begin{quote}
    with the Invariant Sections being LIST THEIR TITLES, with the
    Front-Cover Texts being LIST, and with the Back-Cover Texts being LIST.
\end{quote}
\bigskip

If you have Invariant Sections without Cover Texts, or some other
combination of the three, merge those two alternatives to suit the
situation.

If your document contains nontrivial examples of program code, we
recommend releasing these examples in parallel under your choice of
free software license, such as the GNU General Public License,
to permit their use in free software.




% ############################################################################
\documentclass{howto}
\title{Candygram}
\makeindex
\date{\today}
\author{Michael Hobbs}
\authoraddress{
        Email: \email{mike@hobbshouse.org}
}
\release{1.0 beta 1}
\setshortversion{1.0}

\newcommand{\Erlang}{\ulink{Erlang}{http://www.erlang.org/}}
\newcommand{\erlangbook}{\citetitle[http://www.erlang.org/download/erlang-book-part1.pdf]{Concurrent Programming in Erlang}}

% PDF output doesn't display some symbols well, unless it is in math mode.
% The math mode graphics generated by latex2html, on the other hand, look awful.
\ifpdf
 	\newcommand{\lessthan}[0]{\begin{math}<\end{math}}
 	\newcommand{\greaterthan}[0]{\begin{math}>\end{math}}
 	\newcommand{\pipe}[0]{\begin{math}|\end{math}}
\else
 	\newcommand{\lessthan}[0]{<}
 	\newcommand{\greaterthan}[0]{>}
 	\newcommand{\pipe}[0]{|}
\fi



\begin{document}

\maketitle

Copyright \copyright\ 2004 Michael Hobbs.

Permission is granted to copy, distribute and/or modify this document
under the terms of the GNU Free Documentation License, Version 1.2
or any later version published by the Free Software Foundation;
with no Invariant Sections, no Front-Cover Texts, and no Back-Cover Texts.
A copy of the license is included in the section entitled ``GNU
Free Documentation License''.

\begin{abstract}
\noindent
Candygram is a Python implementation of \Erlang\ concurrency primitives. Erlang
is widely respected for its elegant built-in facilities for concurrent
programming. This package attempts to emulate those facilities as closely as
possible in Python. With Candygram, developers can send and receive messages
between threads using semantics nearly identical to those in the Erlang
language.
\end{abstract}

\tableofcontents



% ############################################################################
\section{Download And Install}

You can download all available versions of Candygram from
\ulink{SourceForge.net}
	{http://sourceforge.net/project/showfiles.php?group_id=114295}. The current
stable release, as of this writing, is \version. You can also download this
document in PDF format from the link above.

To install, uncompress the zip or tar file, \program{cd} into the directory, and
run:
\begin{verbatim}
python setup.py install
\end{verbatim}

Windows users can just download the installer program and run it.



% ############################################################################
\section{Overview}

\begin{notice}[note]
Erlang uses a peculiar terminology in respect to threads. In Erlang parlance,
threads are called ``processes''. This terminology is due to a couple of
reasons. First, the Erlang runtime environment is a virtual machine. Second,
because Erlang is a functional language, no state is shared among its threads.
Since the concurrent tasks in an Erlang system run as peers within a single
[virtual] machine and don't share state, they are therefore named ``Processes''.

To avoid confusion when using this document alongside the Erlang documentation,
the remainder of this document uses the word ``process'' instead of ``thread''.
You are free, however, to pronounce the word ``process'' however you wish.
\end{notice}

This package provides an implementation of the following \Erlang\ core
functions:
\begin{itemize}
	\item \function{spawn()}
	\item \function{send()}
	\item \keyword{receive}
	\item \function{link()}
\end{itemize}

These 4 functions form the core of Erlang's concurrency services. The
\function{spawn()} function creates a new process, the \function{send()}
function sends a message to another process, the \keyword{receive} statement
specifies what to do with received messages, and the \function{link()} function
allows one process to monitor the status of another process. In addition to
these core functions, this package also provides implementations of several
supplemental functions such as \function{spawn_link()} and \function{exit()}.

The beauty of the Erlang system is that it is simple and yet powerful. To
communicate with another process, you simply send a message to it. You do not
need not worry about locks, semaphores, mutexes, etc. to share information among
concurrent tasks. Developers mostly use message passing only to implement the
producer/consumer model. When you combine message passing with the flexibility
of the \keyword{receive} statement, however, it becomes much more powerful. For
example, by using timeouts and receive patterns, a process may easily handle its
messages as a state machine, or as a priority queue.

For those who wish to become more familiar with Erlang, \erlangbook\ provides a
very complete introduction. In particular, this package implements all of the
functions described in chapter 5 and sections 7.2, 7.3, and 7.5 of that book.



% ############################################################################
\section{Receiver Patterns}
\label{patterns}
\index{patterns}

\newcommand{\addhandler}{\method{addHandler()}}

Erlang provides pattern matching in its very syntax, which the \keyword{receive}
statement uses to its advantage. Since Python does not provide pattern matching
in its syntax, we must use a slightly different mechanism to match messages. The
first parameter passed to the \method{Receiver.addHandler()} method
[\ref{Receiver}] can be any Python value. The \addhandler\ method uses this
value as a pattern and interprets it in the following way:
\begin{enumerate}
	\item If the value is \constant{candygram.Any}, then any message will match.
	\item If the value is a type object or a class, then any message that
		\function{isinstance()} of the type or class will match.
	\item If the value is \function{callable()}, then a message will match if the
		function/method returns \constant{True} when called with the message.
	\item If the value is a list or tuple, then a message will match only if it
		is a list or tuple, respectively, and of the same length. Also, each value
		in the sequence is used as a pattern and the sequence as a whole will
		match only if every pattern in the sequence matches its associated value in
		the message.
	\item If the value is a list or tuple, and the last item in the sequence is
		\constant{candygram.AnyRemaining}, then the above rule applies except that
		the message sequence may be any length that is \greaterthan= len(pattern)-1.
	\item If the value is a dictionary, then a message will match only if it
		is a dictionary that contains all of the same keys as the pattern value.
		Also, each value in the dictionary is used as a pattern and the
		dictionary as a whole will match only if every pattern in the dictionary
		matches its associated value in the message.
	\item Lastly, \addhandler\ treats any other value as a literal pattern. That
		is, a message will match if it is equal to the given value.
\end{enumerate}

\subsection{Examples}
Let's illustrate these rules by example. In the table below, the first column
contains a Python value that is used as a pattern. The second column contains
Python values that match the pattern and the third column contains Python values
that do not match the pattern.
\begin{tableiii}{l|l|l}{textrm}{Pattern}{Matches}{Non-Matches}
\lineiii{Any}
	{'shark', 13.7, (1, '', lambda: true)}
	{}
\lineiii{'shark'}
	{'shark'}
	{'dolphin', 42, []}
\lineiii{13.7}
	{13.7}
	{'shark', 13.6, \{'A': 14\}}
\lineiii{int}
	{13, 42, 0}
	{'shark', 13.7, []}
\lineiii{str}
	{'shark', '', 'dolphin'}
	{42, 0.9, lambda: True}
\lineiii{lambda x: x \greaterthan\ 20}
	{42, 100, 67.7}
	{13, 0, -67.7}
\lineiii{(str, int)}
	{('shark', 42), ('dolphin', 0)}
	{['shark', 42], ('dolphin', 42, 0)}
\lineiii{(str, int, AnyRemaining)}
	{('dolphin', 0), ('dolphin', 42, 0.9)}
	{('dolphin',), (42, 'dolphin')}
\lineiii{[str, 20, lambda x: x \lessthan\ 0]}
	{['shark', 20, -54.76], ['dolphin', 20, -1]}
	{['shark', 21, -6], [20, 20, -1], ['', 20]}
\lineiii{\{'S': int, 19: str\}}
	{\{'S': 3, 19: 'foo'\}, \{'S': -65, 19: 'bar', 'T': 'me'\}}
	{\{'S': 'Charlie', 19: 'foo'\}, \{'S': 3\}}
\end{tableiii}



% ############################################################################
\section{The \module{candygram} module}

\declaremodule{extension}{candygram}
\modulesynopsis{Erlang concurrency primitives}

The \module{candygram} module exports the following functions, classes,
constants, and exceptions. Since the name \module{candygram} is a bit long, you
would typically import the module in one of the following ways:
\begin{verbatim}
>>> from candygram import *
\end{verbatim}
or
\begin{verbatim}
>>> import candygram as cg
\end{verbatim}



% ----------------------------------------------------------------------------
\subsection{Functions}

\begin{funcdesc}{spawn}{func\optional{, args\moreargs}}
Create a new concurrent process by calling the function \var{func} with the
\var{args} argument list and return the resulting \class{Process} instance.
When the function returns, the process terminates. Raises a \code{'badarg'}
\exception{ExitError} if \var{func} is not \function{callable()}.
\end{funcdesc}

\begin{funcdesc}{link}{proc}
Create a link to the process \var{proc}, if there is not such a link already. If
a process attempts to create a link to itself, nothing is done. Raises a
\code{'badarg'} \exception{ExitError} if \var{proc} is not a \class{Process}
instance. Sends a \code{'noproc'} \code{'EXIT'} signal to calling process if the
\var{proc} process is no longer alive.

When a process terminates, it sends an \code{'EXIT'} signal is to all of its
linked processes. If a process terminates normally, it sends a \code{'normal'}
\code{'EXIT'} signal to its linked processes.

All links are bidirectional. That is, if process A calls \code{link(B)}, then if
process B terminates, it sends an \code{'EXIT'} signal to process A. Conversely,
if process A terminates, it likewise sends an \code{'EXIT'} signal to process B.

Refer to the \function{processFlag()} function for details about handling
signals.
\end{funcdesc}

\begin{funcdesc}{spawnLink}{func\optional{, args\moreargs}}
This function is identical to the following code being evaluated in an atomic
operation:
\begin{verbatim}
>>> proc = spawn(func, args...)
>>> link(proc)
\end{verbatim}
This function is necessary since the process created might run immediately and
fail before \function{link()} is called. Returns the \class{Process} instance of
the newly created process. Raises a \code{'badarg'} \exception{ExitError} if
\var{func} is not \function{callable()}.
\end{funcdesc}

\begin{funcdesc}{unlink}{proc}
Remove the link, if there is one, from the calling process to another process
given by the \var{proc} argument. The function does not fail if the calling
process is not linked to \var{proc}, or if \var{proc} is not alive. Raises a
\code{'badarg'} \exception{ExitError} if \var{proc} is not a \class{Process}
instance.
\end{funcdesc}

\begin{funcdesc}{isProcessAlive}{proc}
Return \constant{True} if the process is alive, i.e., has not terminated.
Otherwise, return \constant{False}. Raises a \code{'badarg'}
\exception{ExitError} if \var{proc} is not a \class{Process} instance.
\end{funcdesc}

\begin{funcdesc}{self}{}
Return the \class{Process} instance of the calling process.
\end{funcdesc}

\begin{funcdesc}{self_}{}
An alias for the \function{self()} function. You can use this function in class
methods where \var{self} is already defined.
\end{funcdesc}

\begin{funcdesc}{processes}{}
Return a list of all active processes.
\end{funcdesc}

\begin{funcdesc}{send}{proc, message}
Send the \var{message} to the \var{proc} process and return \var{message}. This
is the same as \var{proc}\code{.send(}\var{message}\code{)}. Raises a
\code{'badarg'} \exception{ExitError} if \var{proc} is not a \class{Process}
instance.
\end{funcdesc}

\begin{funcdesc}{exit}{\optional{proc, }reason}
When the \var{proc} argument is not given, this function raises an
\exception{ExitError} with the reason \var{reason}. \var{reason} can be any
value.

When the \var{proc} argument is given, this function sends an \code{'EXIT'}
signal to the process \var{proc}. This is not necessarily the same as sending an
\code{'EXIT'} message to \var{proc}. They are the same if \var{proc} is trapping
exits. If \var{proc} is not trapping exits, however, the \var{proc} process
terminates and propagates the \code{'EXIT'} signal in turn to its linked
processes.

If the \var{reason} is the string \code{'kill'}, for example
\code{exit(proc, 'kill')}, an untrappable \code{'EXIT'} signal is sent to the
process. In other words, the \var{proc} process is unconditionally killed.

Refer to the \function{processFlag()} function for details about trapping exits.
\end{funcdesc}

\begin{funcdesc}{processFlag}{flag, option}
Set the given \var{flag} for the calling process. Returns the old value of the
flag. Raises a \code{'badarg'} \exception{ExitError} if \var{flag} is not a
recognized flag value, or if \var{option} is not a recognized value for
\var{flag}.

Currently, \code{'trap_exit'} is the only recognized flag value. When
\code{'trap_exit'} is set to \constant{True}, \code{'EXIT'} signals arriving to
a process are converted to \code{('EXIT', from, reason)} messages, which can be
received as ordinary messages. If \code{'trap_exit'} is set to \constant{False},
the process exits if it receives an \code{'EXIT'} signal other than
\code{'normal'} and propagates the \code{'EXIT'} signal to its linked processes.
Application processes should normally not trap exits.
\end{funcdesc}



% ----------------------------------------------------------------------------
\subsection{Process Objects}

\begin{classdesc}{Process}{}
Represents a concurrent process. A \class{Process} is never created via its
constructor. The \function{spawn()} and \function{spawnLink()} functions create
all processes.

\begin{methoddesc}{isProcessAlive}{}
Return \constant{True} if the process is alive, i.e., has not terminated.
Otherwise, return \constant{False}.
\end{methoddesc}

\begin{methoddesc}{isAlive}{}
An alias for the \method{isProcessAlive()} method. (The word ``process'' is
redundant in a method name when the method is a member of the \class{Process}
class.)
\end{methoddesc}

\begin{methoddesc}{send}{message}
Send the \var{message} to this process and return \var{message}. A routine
running in a separate process calls this method to place the given
\var{message} into this process's mailbox. A \class{Receiver} that is operating
in this process may then pick up the message.

Sending a message is an asynchronous operation so the \method{send()} call does
not wait for a \class{Receiver} to retrieve the message. Even if this process
has already terminated, the system does not notify the sender. Messages are
always delivered, and always in the same order they were sent.
\end{methoddesc}

\begin{methoddesc}{__or__}{message}
\opindex{|}
An alias for the \method{send()} method. The OR operator, `\pipe', is aliased to
the \method{send()} method so that developers can use a more Erlangy syntax to
send messages. In Erlang, the `!' primitive sends messages. For example:
\begin{verbatim}
>>> proc | ('knock-knock', 'candygram')
\end{verbatim}
\end{methoddesc}

\end{classdesc}



% ----------------------------------------------------------------------------
\subsection{Receiver Objects}
\label{Receiver}

\begin{classdesc}{Receiver}{}
Retrieves messages from a process's mailbox. Every process maintains a mailbox,
which contains messages that have been sent to it via the \method{send()}
method. To retrieve the messages out of the mailbox, a process must use a
\class{Receiver} object. Like the \keyword{receive} statement in Erlang, a
\class{Receiver} object compares an incoming message against multiple patterns
[\ref{patterns}] and invokes the callback function that is associated
with the first matching pattern.

\begin{notice}[warning]
You may not use a single \class{Receiver} instance across multiple processes.
The \class{Receiver} raises an \exception{AssertionError} if you call one
of its methods from a process other than the one that created it. You should
take care, therefore, to create a \class{Receiver} within the process that will
be using it.
\end{notice}

\begin{methoddesc}{addHandler}{pattern\optional{, func\optional{, args\moreargs}}}
Register a handler function \var{func} for the \var{pattern}. The
\method{receive()} method calls the \var{func} with the \var{args} argument
list when it receives a message that matches \var{pattern}. When a handler
function \var{func} is not specified, the \method{receive()} method removes any
matching message from the mailbox and does nothing more with it. Refer to
section \ref{patterns} for details about how to specify message patterns. Raises
a \code{'badarg'} \exception{ExitError} if \var{func} is not
\function{callable()}.

If any of the \var{args} parameters is \constant{candygram.Message}, then the
\method{receive()} method replaces that parameter with the matching message when
it invokes \var{func}.

Refer to the \method{receive()} documentation for details about the
mechanism by which it invokes the handlers.
\end{methoddesc}

\begin{methoddesc}{__setitem__}{pattern, funcWithArgs}
\opindex{[]}
An alias for the \method{addHandler()} method. If \var{funcWithArgs} is a tuple,
then this method sends the first element in the tuple as the \var{func}
parameter to \method{addHandler()} and the remaining as the \var{args}. If
\var{funcWithArgs} is not a tuple, then this method assumes it to be a handler
function and sends it as the \var{func} parameter to \method{addHandler()}.

\method{__setitem__()} is an alias to the \method{addHandler()} method so that
developers can use a more Erlangy syntax to specify handlers. In Erlang, the
``\var{pattern} -\greaterthan\ \var{func}'' syntax specifies pattern guards. For
example:
\begin{verbatim}
>>> r = Receiver()
>>> r['knock-knock', 'candygram'] = answer, 'From whom?'
\end{verbatim}
\end{methoddesc}

\begin{methoddesc}{__getitem__}{pattern}
\opindex{[]}
An alias for the \method{addHandler()} method. This method calls
\method{addHandler()} without a \var{func} parameter.

\method{__getitem__()} is an alias to the \method{addHandler()} method so that
developers can use a more Erlangy syntax to specify handlers. In Erlang, the
``\var{pattern} -\greaterthan\ \var{func}'' syntax specifies pattern guards. For
example:
\begin{verbatim}
>>> r = Receiver()
>>> r['knock-knock', 'dolphin']  # ignore dolphins
\end{verbatim}
\end{methoddesc}

\begin{methoddesc}{addHandlers}{receiver}
Register all handler functions in \var{receiver} with this \class{Receiver}
object. This method adds all of the patterns and handler functions that have
been added to the given \var{receiver} to this receiver, in the same order. You
can use this method to make copies of \class{Receiver} objects. You may use a
\var{receiver} that was created in a different process. Raises a \code{'badarg'}
\exception{ExitError} if \var{receiver} is not a \class{Receiver} instance.
\end{methoddesc}

\begin{methoddesc}{after}{timeout\optional{, func\optional{, args\moreargs}}}
Register a timeout handler function \var{func}. The \method{receive()} method
calls the \var{func} with the \var{args} argument list when it does not receive
a matching message within \var{timeout} milliseconds. Raises a \code{'badarg'}
\exception{ExitError} if \var{func} is not \function{callable()} or if
\var{timeout} is not an integer. Raises an \exception{AssertionError} if the
\method{after()} method has already been invoked.

If the \method{receive()} method does not find a matching message within
\var{timeout} milliseconds, it invokes \var{func} with \var{args} and returns
the result. When a handler function \var{func} is not specified, the
\method{receive()} method returns \constant{None} if it times out.

None of the \var{args} parameters should be \constant{candygram.Message}, since
there is no message to pass when a timeout occurs.
\end{methoddesc}

\begin{methoddesc}{receive}{\optional{timeout\optional{, func\optional{, args\moreargs}}}}
Find a matching message in the process's mailbox, invoke the related handler
function, and return the result. The \class{Receiver} object compares the first
message in the process's mailbox with each of its registered patterns, in the
order that the patterns were added via the \method{addHandler()} method. If any
pattern matches the message, this method removes the message from the mailbox,
calls the associated handler function, and returns its result. If the matching
pattern does not have a handler function associated with it, this method returns
\constant{None}. If no pattern matches the message, this method will leave the
message in the mailbox, skips to the next message in the mailbox, and compares
it with each of the patterns. It continues on through each of the messages in
the mailbox until a match is found.

If no message matches any of the patterns, or if the mailbox is empty, this
method blocks until the process receives a message that does match one of the
patterns, or until the given \var{timeout} has elapsed. If no \var{timeout} is
specified, this method blocks indefinitely. If a \var{timeout} is specified, it
is equivalent to calling \method{after(\var{timeout}, \var{func}, \var{args})}
just prior to \method{receive()}. This method raises a \code{'badarg'}
\exception{ExitError}, therefore, if \var{func} is not \function{callable()} or
if \var{timeout} is not an integer. It also raises an \exception{AssertionError}
if a \var{timeout} is specified and the \method{after()} method has already been
invoked.
\end{methoddesc}

\begin{methoddesc}{__call__}{\optional{timeout\optional{, func\optional{, args\moreargs}}}}
\opindex{()}
An alias for the \method{receive()} method. \method{__call__()} is an alias to
the \method{receive()} method so that developers can use a shortened, more
Erlangy syntax with \class{Receiver}s. For example:
\begin{verbatim}
>>> def convert():
...     r = Receiver()
...     r['one'] = lambda: 1
...     r['two'] = lambda: 2
...     r['three'] = lambda: 3
...     return r()
\end{verbatim}
\end{methoddesc}

\begin{methoddesc}{__iter__}{}
Return an iterator that repeatedly invokes \method{receive()}. Candygram code
often uses the following idiom:
\begin{verbatim}
... while True:
...     result = receiver.receive()
...     # do whatever with the result...
\end{verbatim}
With an iterator, you can spell the code above like this instead:
\begin{verbatim}
... for result in receiver:
...     # do whatever with the result...
\end{verbatim}
\end{methoddesc}

\end{classdesc}



% ----------------------------------------------------------------------------
\subsection{Constants}

\begin{datadesc}{Any}
When used in a pattern, the \constant{Any} constant will match any value.
\end{datadesc}

\begin{datadesc}{AnyRemaining}
When used in the last position of a tuple or list pattern, the
\constant{AnyRemaining} constant will cause the sequence to match a sequence of
the same kind (tuple or list) that has a length \greaterthan= len(pattern)-1.
The patterns in the sequence prior to \constant{AnyRemaining} must still match
their respective values in a message for the whole sequence to match.
\end{datadesc}

\begin{datadesc}{Message}
When used as a function parameter in \method{Receiver.addHandler()}, the
\method{Receiver.receive()} method replaces the parameter with the matching
message when it invokes the handler function.
\end{datadesc}



% ----------------------------------------------------------------------------
\subsection{Exceptions}

\begin{excdesc}{ExitError}
Represents \code{'EXIT'} errors from Erlang. If an Erlang function can
cause a failure under certain circumstances, then the corresponding Candygram
function raises an \exception{ExitError} under the same circumstances.

A process also raises an \exception{ExitError} in all of its linked processes
if it terminates for a reason other than \code{'normal'}.

\begin{notice}[note]
When a process terminates with a reason other than \code{'normal'}, it does not
immediately raise an \exception{ExitError} in all linked processes. Candygram
defers the \exception{ExitError} instead, and raises it the next time one of its
functions or methods is called. (Python does not allow you to unconditionally
interrupt a separate thread.)
\end{notice}

\begin{memberdesc}{reason}
Specifies the reason why the \exception{ExitError} was raised. Can be any value.
\end{memberdesc}

\begin{memberdesc}{proc}
The process that originally caused the \exception{ExitError}.
\end{memberdesc}

\end{excdesc}



% ############################################################################
\section{Examples}
There is a directory named \file{examples} within every distribution of
Candygram. This directory contains the Candygram equivalents of all the sample
programs found in chapter 5 and sections 7.2, 7.3, and 7.5 of \erlangbook. The
files named \file{program_X.X.py} are direct translations of the Erlang
programs. The files named \file{program_X.X_alt.py} are alternate, more liberal
translations that use more Pythonic idioms. \note{Since the CPython interpreter
does not perform tail-call optimization, the tail-recursive style of the direct
translations is not a recommended practice.}

The \file{examples} directory also contains a few other modules that demonstrate
how you can Candygram to perform some handy functions.

If you are not already familiar with Erlang, the best way to become familiar
with Candygram is to read chapter 5 in \erlangbook\ and follow along using the
example programs located in the \file{examples} directory. If you are already
familiar with Erlang, here are a few code snippets to give you a taste of
Candygram.

\begin{verbatim}
>>> from candygram import *
>>> def proc_func():
...     r = Receiver()
...     r['land shark'] = lambda m: 'Go Away ' + m, Message
...     r['candygram'] = lambda m: 'Hello ' + m, Message
...     for message in r:
...         print message
...
>>> proc = spawn(proc_func)
>>> proc | 'land shark'
>>> proc | 'candygram'
\end{verbatim}
Running the code above produces the following output:
\begin{verbatim}
Go Away land shark
Hello candygram
\end{verbatim}

The code above uses a rather Erlangy syntax. Here is a more Pythonic version
that does the same:
\begin{verbatim}
>>> import candygram as cg
>>> def proc_func():
...     r = cg.Receiver()
...     r.addHandler('land shark', shut_door, cg.Message)
...     r.addHandler('candygram', open_door, cg.Message)
...     for message in r:
...         print message
...
>>> def shut_door(name):
...     return 'Go Away ' + name
...
>>> def open_door(name):
...     return 'Hello ' + name
...
>>> proc = cg.spawn(proc_func)
>>> proc.send('land shark')
>>> proc.send('candygram')
\end{verbatim}

Lastly, here is an example with more elaborate patterns:
\begin{verbatim}
>>> from candygram import *
>>> def proc_func(name):
...     r = Receiver()
...     r['msg', Process, str] = print_string, name, Message
...     r['msg', Process, str, Any] = print_any, name, Message
...     r[Any]  # Ignore any other messages
...     for result in r:
...         pass
...
>>> def print_string(name, message):
...     msg, process, string = message
...     # 'msg' and 'process' are unused
...     print '%s received: %s' % (name, string)
...
>>> def print_any(name, message):
...     msg, process, prefix, value = message
...     # 'msg' and 'process' are unused
...     print '%s received: %s %s' % (name, prefix, value)
...
>>> a = spawn(proc_func, 'A')
>>> b = spawn(proc_func, 'B')
>>> a | ('msg', b, 'Girl Scout cookies')
>>> a | 'plumber?'
>>> a | ('msg', b, 'The meaning of life is:', 42)
\end{verbatim}
Running the code above produces the following output:
\begin{verbatim}
A received: Girl Scout cookies
A received: The meaning of life is: 42
\end{verbatim}



% ############################################################################
\section{FAQ}

\subsection{Why is the package called Candygram?}
The name Candygram is actually an acronym for ``the Candygram Acronym Does Not
Yield a Good Reference to Anything Meaningful.''

\subsection{But wait, doesn't that spell {\sc cadnygram}?}
Yes, you are quite observant. In order to form a compromise with the French
acronym, which is {\sc canydgram}, the Candygram committee standardized the
official acronym as {\sc Candygram}.

\subsection{How do you pronounce Candygram?}
This question produces an outrageous amount of heated debate. Some claim that
you pronounce it with short A's, as in tomato, while others claim that you
pronounce it with long A's, as in potato. Both sides, however, are completely
wrong; the correct pronunciation for Candygram is ``throat warbler mangrove.''



% ############################################################################
\section{Feedback}

Please submit all bug reports, feature requests, etc. to the appropriate tracker
on \ulink{SourceForge.net}{http://sourceforge.net/tracker/?group_id=114295}.

General discussion takes place on the
\email{candygram-discuss@lists.sourceforge.net} mailing list. You can subscribe
to this list by visiting \ulink{this page}
	{http://lists.sourceforge.net/lists/listinfo/candygram-discuss}.



% ############################################################################
\appendix
\section{GNU Free Documentation License}
\label{fdl}
 \begin{center}

       Version 1.2, November 2002


 Copyright \copyright 2000,2001,2002  Free Software Foundation, Inc.

 \bigskip

     59 Temple Place, Suite 330, Boston, MA  02111-1307  USA

 \bigskip

 Everyone is permitted to copy and distribute verbatim copies
 of this license document, but changing it is not allowed.
\end{center}


\begin{center}
{\bf\large Preamble}
\end{center}

The purpose of this License is to make a manual, textbook, or other
functional and useful document "free" in the sense of freedom: to
assure everyone the effective freedom to copy and redistribute it,
with or without modifying it, either commercially or noncommercially.
Secondarily, this License preserves for the author and publisher a way
to get credit for their work, while not being considered responsible
for modifications made by others.

This License is a kind of "copyleft", which means that derivative
works of the document must themselves be free in the same sense.  It
complements the GNU General Public License, which is a copyleft
license designed for free software.

We have designed this License in order to use it for manuals for free
software, because free software needs free documentation: a free
program should come with manuals providing the same freedoms that the
software does.  But this License is not limited to software manuals;
it can be used for any textual work, regardless of subject matter or
whether it is published as a printed book.  We recommend this License
principally for works whose purpose is instruction or reference.


\begin{center}
{\Large\bf 1. APPLICABILITY AND DEFINITIONS}
\end{center}

This License applies to any manual or other work, in any medium, that
contains a notice placed by the copyright holder saying it can be
distributed under the terms of this License.  Such a notice grants a
world-wide, royalty-free license, unlimited in duration, to use that
work under the conditions stated herein.  The \textbf{"Document"}, below,
refers to any such manual or work.  Any member of the public is a
licensee, and is addressed as \textbf{"you"}.  You accept the license if you
copy, modify or distribute the work in a way requiring permission
under copyright law.

A \textbf{"Modified Version"} of the Document means any work containing the
Document or a portion of it, either copied verbatim, or with
modifications and/or translated into another language.

A \textbf{"Secondary Section"} is a named appendix or a front-matter section of
the Document that deals exclusively with the relationship of the
publishers or authors of the Document to the Document's overall subject
(or to related matters) and contains nothing that could fall directly
within that overall subject.  (Thus, if the Document is in part a
textbook of mathematics, a Secondary Section may not explain any
mathematics.)  The relationship could be a matter of historical
connection with the subject or with related matters, or of legal,
commercial, philosophical, ethical or political position regarding
them.

The \textbf{"Invariant Sections"} are certain Secondary Sections whose titles
are designated, as being those of Invariant Sections, in the notice
that says that the Document is released under this License.  If a
section does not fit the above definition of Secondary then it is not
allowed to be designated as Invariant.  The Document may contain zero
Invariant Sections.  If the Document does not identify any Invariant
Sections then there are none.

The \textbf{"Cover Texts"} are certain short passages of text that are listed,
as Front-Cover Texts or Back-Cover Texts, in the notice that says that
the Document is released under this License.  A Front-Cover Text may
be at most 5 words, and a Back-Cover Text may be at most 25 words.

A \textbf{"Transparent"} copy of the Document means a machine-readable copy,
represented in a format whose specification is available to the
general public, that is suitable for revising the document
straightforwardly with generic text editors or (for images composed of
pixels) generic paint programs or (for drawings) some widely available
drawing editor, and that is suitable for input to text formatters or
for automatic translation to a variety of formats suitable for input
to text formatters.  A copy made in an otherwise Transparent file
format whose markup, or absence of markup, has been arranged to thwart
or discourage subsequent modification by readers is not Transparent.
An image format is not Transparent if used for any substantial amount
of text.  A copy that is not "Transparent" is called \textbf{"Opaque"}.

Examples of suitable formats for Transparent copies include plain
ASCII without markup, Texinfo input format, LaTeX input format, SGML
or XML using a publicly available DTD, and standard-conforming simple
HTML, PostScript or PDF designed for human modification.  Examples of
transparent image formats include PNG, XCF and JPG.  Opaque formats
include proprietary formats that can be read and edited only by
proprietary word processors, SGML or XML for which the DTD and/or
processing tools are not generally available, and the
machine-generated HTML, PostScript or PDF produced by some word
processors for output purposes only.

The \textbf{"Title Page"} means, for a printed book, the title page itself,
plus such following pages as are needed to hold, legibly, the material
this License requires to appear in the title page.  For works in
formats which do not have any title page as such, "Title Page" means
the text near the most prominent appearance of the work's title,
preceding the beginning of the body of the text.

A section \textbf{"Entitled XYZ"} means a named subunit of the Document whose
title either is precisely XYZ or contains XYZ in parentheses following
text that translates XYZ in another language.  (Here XYZ stands for a
specific section name mentioned below, such as \textbf{"Acknowledgements"},
\textbf{"Dedications"}, \textbf{"Endorsements"}, or \textbf{"History"}.)
To \textbf{"Preserve the Title"}
of such a section when you modify the Document means that it remains a
section "Entitled XYZ" according to this definition.

The Document may include Warranty Disclaimers next to the notice which
states that this License applies to the Document.  These Warranty
Disclaimers are considered to be included by reference in this
License, but only as regards disclaiming warranties: any other
implication that these Warranty Disclaimers may have is void and has
no effect on the meaning of this License.


\begin{center}
{\Large\bf 2. VERBATIM COPYING}
\end{center}

You may copy and distribute the Document in any medium, either
commercially or noncommercially, provided that this License, the
copyright notices, and the license notice saying this License applies
to the Document are reproduced in all copies, and that you add no other
conditions whatsoever to those of this License.  You may not use
technical measures to obstruct or control the reading or further
copying of the copies you make or distribute.  However, you may accept
compensation in exchange for copies.  If you distribute a large enough
number of copies you must also follow the conditions in section 3.

You may also lend copies, under the same conditions stated above, and
you may publicly display copies.


\begin{center}
{\Large\bf 3. COPYING IN QUANTITY}
\end{center}


If you publish printed copies (or copies in media that commonly have
printed covers) of the Document, numbering more than 100, and the
Document's license notice requires Cover Texts, you must enclose the
copies in covers that carry, clearly and legibly, all these Cover
Texts: Front-Cover Texts on the front cover, and Back-Cover Texts on
the back cover.  Both covers must also clearly and legibly identify
you as the publisher of these copies.  The front cover must present
the full title with all words of the title equally prominent and
visible.  You may add other material on the covers in addition.
Copying with changes limited to the covers, as long as they preserve
the title of the Document and satisfy these conditions, can be treated
as verbatim copying in other respects.

If the required texts for either cover are too voluminous to fit
legibly, you should put the first ones listed (as many as fit
reasonably) on the actual cover, and continue the rest onto adjacent
pages.

If you publish or distribute Opaque copies of the Document numbering
more than 100, you must either include a machine-readable Transparent
copy along with each Opaque copy, or state in or with each Opaque copy
a computer-network location from which the general network-using
public has access to download using public-standard network protocols
a complete Transparent copy of the Document, free of added material.
If you use the latter option, you must take reasonably prudent steps,
when you begin distribution of Opaque copies in quantity, to ensure
that this Transparent copy will remain thus accessible at the stated
location until at least one year after the last time you distribute an
Opaque copy (directly or through your agents or retailers) of that
edition to the public.

It is requested, but not required, that you contact the authors of the
Document well before redistributing any large number of copies, to give
them a chance to provide you with an updated version of the Document.


\begin{center}
{\Large\bf 4. MODIFICATIONS}
\end{center}

You may copy and distribute a Modified Version of the Document under
the conditions of sections 2 and 3 above, provided that you release
the Modified Version under precisely this License, with the Modified
Version filling the role of the Document, thus licensing distribution
and modification of the Modified Version to whoever possesses a copy
of it.  In addition, you must do these things in the Modified Version:

\begin{itemize}
\item[A.]
   Use in the Title Page (and on the covers, if any) a title distinct
   from that of the Document, and from those of previous versions
   (which should, if there were any, be listed in the History section
   of the Document).  You may use the same title as a previous version
   if the original publisher of that version gives permission.

\item[B.]
   List on the Title Page, as authors, one or more persons or entities
   responsible for authorship of the modifications in the Modified
   Version, together with at least five of the principal authors of the
   Document (all of its principal authors, if it has fewer than five),
   unless they release you from this requirement.

\item[C.]
   State on the Title page the name of the publisher of the
   Modified Version, as the publisher.

\item[D.]
   Preserve all the copyright notices of the Document.

\item[E.]
   Add an appropriate copyright notice for your modifications
   adjacent to the other copyright notices.

\item[F.]
   Include, immediately after the copyright notices, a license notice
   giving the public permission to use the Modified Version under the
   terms of this License, in the form shown in the Addendum below.

\item[G.]
   Preserve in that license notice the full lists of Invariant Sections
   and required Cover Texts given in the Document's license notice.

\item[H.]
   Include an unaltered copy of this License.

\item[I.]
   Preserve the section Entitled "History", Preserve its Title, and add
   to it an item stating at least the title, year, new authors, and
   publisher of the Modified Version as given on the Title Page.  If
   there is no section Entitled "History" in the Document, create one
   stating the title, year, authors, and publisher of the Document as
   given on its Title Page, then add an item describing the Modified
   Version as stated in the previous sentence.

\item[J.]
   Preserve the network location, if any, given in the Document for
   public access to a Transparent copy of the Document, and likewise
   the network locations given in the Document for previous versions
   it was based on.  These may be placed in the "History" section.
   You may omit a network location for a work that was published at
   least four years before the Document itself, or if the original
   publisher of the version it refers to gives permission.

\item[K.]
   For any section Entitled "Acknowledgements" or "Dedications",
   Preserve the Title of the section, and preserve in the section all
   the substance and tone of each of the contributor acknowledgements
   and/or dedications given therein.

\item[L.]
   Preserve all the Invariant Sections of the Document,
   unaltered in their text and in their titles.  Section numbers
   or the equivalent are not considered part of the section titles.

\item[M.]
   Delete any section Entitled "Endorsements".  Such a section
   may not be included in the Modified Version.

\item[N.]
   Do not retitle any existing section to be Entitled "Endorsements"
   or to conflict in title with any Invariant Section.

\item[O.]
   Preserve any Warranty Disclaimers.
\end{itemize}

If the Modified Version includes new front-matter sections or
appendices that qualify as Secondary Sections and contain no material
copied from the Document, you may at your option designate some or all
of these sections as invariant.  To do this, add their titles to the
list of Invariant Sections in the Modified Version's license notice.
These titles must be distinct from any other section titles.

You may add a section Entitled "Endorsements", provided it contains
nothing but endorsements of your Modified Version by various
parties--for example, statements of peer review or that the text has
been approved by an organization as the authoritative definition of a
standard.

You may add a passage of up to five words as a Front-Cover Text, and a
passage of up to 25 words as a Back-Cover Text, to the end of the list
of Cover Texts in the Modified Version.  Only one passage of
Front-Cover Text and one of Back-Cover Text may be added by (or
through arrangements made by) any one entity.  If the Document already
includes a cover text for the same cover, previously added by you or
by arrangement made by the same entity you are acting on behalf of,
you may not add another; but you may replace the old one, on explicit
permission from the previous publisher that added the old one.

The author(s) and publisher(s) of the Document do not by this License
give permission to use their names for publicity for or to assert or
imply endorsement of any Modified Version.


\begin{center}
{\Large\bf 5. COMBINING DOCUMENTS}
\end{center}


You may combine the Document with other documents released under this
License, under the terms defined in section 4 above for modified
versions, provided that you include in the combination all of the
Invariant Sections of all of the original documents, unmodified, and
list them all as Invariant Sections of your combined work in its
license notice, and that you preserve all their Warranty Disclaimers.

The combined work need only contain one copy of this License, and
multiple identical Invariant Sections may be replaced with a single
copy.  If there are multiple Invariant Sections with the same name but
different contents, make the title of each such section unique by
adding at the end of it, in parentheses, the name of the original
author or publisher of that section if known, or else a unique number.
Make the same adjustment to the section titles in the list of
Invariant Sections in the license notice of the combined work.

In the combination, you must combine any sections Entitled "History"
in the various original documents, forming one section Entitled
"History"; likewise combine any sections Entitled "Acknowledgements",
and any sections Entitled "Dedications".  You must delete all sections
Entitled "Endorsements".

\begin{center}
{\Large\bf 6. COLLECTIONS OF DOCUMENTS}
\end{center}

You may make a collection consisting of the Document and other documents
released under this License, and replace the individual copies of this
License in the various documents with a single copy that is included in
the collection, provided that you follow the rules of this License for
verbatim copying of each of the documents in all other respects.

You may extract a single document from such a collection, and distribute
it individually under this License, provided you insert a copy of this
License into the extracted document, and follow this License in all
other respects regarding verbatim copying of that document.


\begin{center}
{\Large\bf 7. AGGREGATION WITH INDEPENDENT WORKS}
\end{center}


A compilation of the Document or its derivatives with other separate
and independent documents or works, in or on a volume of a storage or
distribution medium, is called an "aggregate" if the copyright
resulting from the compilation is not used to limit the legal rights
of the compilation's users beyond what the individual works permit.
When the Document is included in an aggregate, this License does not
apply to the other works in the aggregate which are not themselves
derivative works of the Document.

If the Cover Text requirement of section 3 is applicable to these
copies of the Document, then if the Document is less than one half of
the entire aggregate, the Document's Cover Texts may be placed on
covers that bracket the Document within the aggregate, or the
electronic equivalent of covers if the Document is in electronic form.
Otherwise they must appear on printed covers that bracket the whole
aggregate.


\begin{center}
{\Large\bf 8. TRANSLATION}
\end{center}


Translation is considered a kind of modification, so you may
distribute translations of the Document under the terms of section 4.
Replacing Invariant Sections with translations requires special
permission from their copyright holders, but you may include
translations of some or all Invariant Sections in addition to the
original versions of these Invariant Sections.  You may include a
translation of this License, and all the license notices in the
Document, and any Warranty Disclaimers, provided that you also include
the original English version of this License and the original versions
of those notices and disclaimers.  In case of a disagreement between
the translation and the original version of this License or a notice
or disclaimer, the original version will prevail.

If a section in the Document is Entitled "Acknowledgements",
"Dedications", or "History", the requirement (section 4) to Preserve
its Title (section 1) will typically require changing the actual
title.


\begin{center}
{\Large\bf 9. TERMINATION}
\end{center}


You may not copy, modify, sublicense, or distribute the Document except
as expressly provided for under this License.  Any other attempt to
copy, modify, sublicense or distribute the Document is void, and will
automatically terminate your rights under this License.  However,
parties who have received copies, or rights, from you under this
License will not have their licenses terminated so long as such
parties remain in full compliance.


\begin{center}
{\Large\bf 10. FUTURE REVISIONS OF THIS LICENSE}
\end{center}


The Free Software Foundation may publish new, revised versions
of the GNU Free Documentation License from time to time.  Such new
versions will be similar in spirit to the present version, but may
differ in detail to address new problems or concerns.  See
http://www.gnu.org/copyleft/.

Each version of the License is given a distinguishing version number.
If the Document specifies that a particular numbered version of this
License "or any later version" applies to it, you have the option of
following the terms and conditions either of that specified version or
of any later version that has been published (not as a draft) by the
Free Software Foundation.  If the Document does not specify a version
number of this License, you may choose any version ever published (not
as a draft) by the Free Software Foundation.


\begin{center}
{\Large\bf ADDENDUM: How to use this License for your documents}
\end{center}

To use this License in a document you have written, include a copy of
the License in the document and put the following copyright and
license notices just after the title page:

\bigskip
\begin{quote}
    Copyright \copyright  YEAR  YOUR NAME.
    Permission is granted to copy, distribute and/or modify this document
    under the terms of the GNU Free Documentation License, Version 1.2
    or any later version published by the Free Software Foundation;
    with no Invariant Sections, no Front-Cover Texts, and no Back-Cover Texts.
    A copy of the license is included in the section entitled "GNU
    Free Documentation License".
\end{quote}
\bigskip

If you have Invariant Sections, Front-Cover Texts and Back-Cover Texts,
replace the "with...Texts." line with this:

\bigskip
\begin{quote}
    with the Invariant Sections being LIST THEIR TITLES, with the
    Front-Cover Texts being LIST, and with the Back-Cover Texts being LIST.
\end{quote}
\bigskip

If you have Invariant Sections without Cover Texts, or some other
combination of the three, merge those two alternatives to suit the
situation.

If your document contains nontrivial examples of program code, we
recommend releasing these examples in parallel under your choice of
free software license, such as the GNU General Public License,
to permit their use in free software.




% ############################################################################
\documentclass{howto}
\title{Candygram}
\makeindex
\date{\today}
\author{Michael Hobbs}
\authoraddress{
        Email: \email{mike@hobbshouse.org}
}
\release{1.0 beta 1}
\setshortversion{1.0}

\newcommand{\Erlang}{\ulink{Erlang}{http://www.erlang.org/}}
\newcommand{\erlangbook}{\citetitle[http://www.erlang.org/download/erlang-book-part1.pdf]{Concurrent Programming in Erlang}}

% PDF output doesn't display some symbols well, unless it is in math mode.
% The math mode graphics generated by latex2html, on the other hand, look awful.
\ifpdf
 	\newcommand{\lessthan}[0]{\begin{math}<\end{math}}
 	\newcommand{\greaterthan}[0]{\begin{math}>\end{math}}
 	\newcommand{\pipe}[0]{\begin{math}|\end{math}}
\else
 	\newcommand{\lessthan}[0]{<}
 	\newcommand{\greaterthan}[0]{>}
 	\newcommand{\pipe}[0]{|}
\fi



\begin{document}

\maketitle

Copyright \copyright\ 2004 Michael Hobbs.

Permission is granted to copy, distribute and/or modify this document
under the terms of the GNU Free Documentation License, Version 1.2
or any later version published by the Free Software Foundation;
with no Invariant Sections, no Front-Cover Texts, and no Back-Cover Texts.
A copy of the license is included in the section entitled ``GNU
Free Documentation License''.

\begin{abstract}
\noindent
Candygram is a Python implementation of \Erlang\ concurrency primitives. Erlang
is widely respected for its elegant built-in facilities for concurrent
programming. This package attempts to emulate those facilities as closely as
possible in Python. With Candygram, developers can send and receive messages
between threads using semantics nearly identical to those in the Erlang
language.
\end{abstract}

\tableofcontents



% ############################################################################
\section{Download And Install}

You can download all available versions of Candygram from
\ulink{SourceForge.net}
	{http://sourceforge.net/project/showfiles.php?group_id=114295}. The current
stable release, as of this writing, is \version. You can also download this
document in PDF format from the link above.

To install, uncompress the zip or tar file, \program{cd} into the directory, and
run:
\begin{verbatim}
python setup.py install
\end{verbatim}

Windows users can just download the installer program and run it.



% ############################################################################
\section{Overview}

\begin{notice}[note]
Erlang uses a peculiar terminology in respect to threads. In Erlang parlance,
threads are called ``processes''. This terminology is due to a couple of
reasons. First, the Erlang runtime environment is a virtual machine. Second,
because Erlang is a functional language, no state is shared among its threads.
Since the concurrent tasks in an Erlang system run as peers within a single
[virtual] machine and don't share state, they are therefore named ``Processes''.

To avoid confusion when using this document alongside the Erlang documentation,
the remainder of this document uses the word ``process'' instead of ``thread''.
You are free, however, to pronounce the word ``process'' however you wish.
\end{notice}

This package provides an implementation of the following \Erlang\ core
functions:
\begin{itemize}
	\item \function{spawn()}
	\item \function{send()}
	\item \keyword{receive}
	\item \function{link()}
\end{itemize}

These 4 functions form the core of Erlang's concurrency services. The
\function{spawn()} function creates a new process, the \function{send()}
function sends a message to another process, the \keyword{receive} statement
specifies what to do with received messages, and the \function{link()} function
allows one process to monitor the status of another process. In addition to
these core functions, this package also provides implementations of several
supplemental functions such as \function{spawn_link()} and \function{exit()}.

The beauty of the Erlang system is that it is simple and yet powerful. To
communicate with another process, you simply send a message to it. You do not
need not worry about locks, semaphores, mutexes, etc. to share information among
concurrent tasks. Developers mostly use message passing only to implement the
producer/consumer model. When you combine message passing with the flexibility
of the \keyword{receive} statement, however, it becomes much more powerful. For
example, by using timeouts and receive patterns, a process may easily handle its
messages as a state machine, or as a priority queue.

For those who wish to become more familiar with Erlang, \erlangbook\ provides a
very complete introduction. In particular, this package implements all of the
functions described in chapter 5 and sections 7.2, 7.3, and 7.5 of that book.



% ############################################################################
\section{Receiver Patterns}
\label{patterns}
\index{patterns}

\newcommand{\addhandler}{\method{addHandler()}}

Erlang provides pattern matching in its very syntax, which the \keyword{receive}
statement uses to its advantage. Since Python does not provide pattern matching
in its syntax, we must use a slightly different mechanism to match messages. The
first parameter passed to the \method{Receiver.addHandler()} method
[\ref{Receiver}] can be any Python value. The \addhandler\ method uses this
value as a pattern and interprets it in the following way:
\begin{enumerate}
	\item If the value is \constant{candygram.Any}, then any message will match.
	\item If the value is a type object or a class, then any message that
		\function{isinstance()} of the type or class will match.
	\item If the value is \function{callable()}, then a message will match if the
		function/method returns \constant{True} when called with the message.
	\item If the value is a list or tuple, then a message will match only if it
		is a list or tuple, respectively, and of the same length. Also, each value
		in the sequence is used as a pattern and the sequence as a whole will
		match only if every pattern in the sequence matches its associated value in
		the message.
	\item If the value is a list or tuple, and the last item in the sequence is
		\constant{candygram.AnyRemaining}, then the above rule applies except that
		the message sequence may be any length that is \greaterthan= len(pattern)-1.
	\item If the value is a dictionary, then a message will match only if it
		is a dictionary that contains all of the same keys as the pattern value.
		Also, each value in the dictionary is used as a pattern and the
		dictionary as a whole will match only if every pattern in the dictionary
		matches its associated value in the message.
	\item Lastly, \addhandler\ treats any other value as a literal pattern. That
		is, a message will match if it is equal to the given value.
\end{enumerate}

\subsection{Examples}
Let's illustrate these rules by example. In the table below, the first column
contains a Python value that is used as a pattern. The second column contains
Python values that match the pattern and the third column contains Python values
that do not match the pattern.
\begin{tableiii}{l|l|l}{textrm}{Pattern}{Matches}{Non-Matches}
\lineiii{Any}
	{'shark', 13.7, (1, '', lambda: true)}
	{}
\lineiii{'shark'}
	{'shark'}
	{'dolphin', 42, []}
\lineiii{13.7}
	{13.7}
	{'shark', 13.6, \{'A': 14\}}
\lineiii{int}
	{13, 42, 0}
	{'shark', 13.7, []}
\lineiii{str}
	{'shark', '', 'dolphin'}
	{42, 0.9, lambda: True}
\lineiii{lambda x: x \greaterthan\ 20}
	{42, 100, 67.7}
	{13, 0, -67.7}
\lineiii{(str, int)}
	{('shark', 42), ('dolphin', 0)}
	{['shark', 42], ('dolphin', 42, 0)}
\lineiii{(str, int, AnyRemaining)}
	{('dolphin', 0), ('dolphin', 42, 0.9)}
	{('dolphin',), (42, 'dolphin')}
\lineiii{[str, 20, lambda x: x \lessthan\ 0]}
	{['shark', 20, -54.76], ['dolphin', 20, -1]}
	{['shark', 21, -6], [20, 20, -1], ['', 20]}
\lineiii{\{'S': int, 19: str\}}
	{\{'S': 3, 19: 'foo'\}, \{'S': -65, 19: 'bar', 'T': 'me'\}}
	{\{'S': 'Charlie', 19: 'foo'\}, \{'S': 3\}}
\end{tableiii}



% ############################################################################
\section{The \module{candygram} module}

\declaremodule{extension}{candygram}
\modulesynopsis{Erlang concurrency primitives}

The \module{candygram} module exports the following functions, classes,
constants, and exceptions. Since the name \module{candygram} is a bit long, you
would typically import the module in one of the following ways:
\begin{verbatim}
>>> from candygram import *
\end{verbatim}
or
\begin{verbatim}
>>> import candygram as cg
\end{verbatim}



% ----------------------------------------------------------------------------
\subsection{Functions}

\begin{funcdesc}{spawn}{func\optional{, args\moreargs}}
Create a new concurrent process by calling the function \var{func} with the
\var{args} argument list and return the resulting \class{Process} instance.
When the function returns, the process terminates. Raises a \code{'badarg'}
\exception{ExitError} if \var{func} is not \function{callable()}.
\end{funcdesc}

\begin{funcdesc}{link}{proc}
Create a link to the process \var{proc}, if there is not such a link already. If
a process attempts to create a link to itself, nothing is done. Raises a
\code{'badarg'} \exception{ExitError} if \var{proc} is not a \class{Process}
instance. Sends a \code{'noproc'} \code{'EXIT'} signal to calling process if the
\var{proc} process is no longer alive.

When a process terminates, it sends an \code{'EXIT'} signal is to all of its
linked processes. If a process terminates normally, it sends a \code{'normal'}
\code{'EXIT'} signal to its linked processes.

All links are bidirectional. That is, if process A calls \code{link(B)}, then if
process B terminates, it sends an \code{'EXIT'} signal to process A. Conversely,
if process A terminates, it likewise sends an \code{'EXIT'} signal to process B.

Refer to the \function{processFlag()} function for details about handling
signals.
\end{funcdesc}

\begin{funcdesc}{spawnLink}{func\optional{, args\moreargs}}
This function is identical to the following code being evaluated in an atomic
operation:
\begin{verbatim}
>>> proc = spawn(func, args...)
>>> link(proc)
\end{verbatim}
This function is necessary since the process created might run immediately and
fail before \function{link()} is called. Returns the \class{Process} instance of
the newly created process. Raises a \code{'badarg'} \exception{ExitError} if
\var{func} is not \function{callable()}.
\end{funcdesc}

\begin{funcdesc}{unlink}{proc}
Remove the link, if there is one, from the calling process to another process
given by the \var{proc} argument. The function does not fail if the calling
process is not linked to \var{proc}, or if \var{proc} is not alive. Raises a
\code{'badarg'} \exception{ExitError} if \var{proc} is not a \class{Process}
instance.
\end{funcdesc}

\begin{funcdesc}{isProcessAlive}{proc}
Return \constant{True} if the process is alive, i.e., has not terminated.
Otherwise, return \constant{False}. Raises a \code{'badarg'}
\exception{ExitError} if \var{proc} is not a \class{Process} instance.
\end{funcdesc}

\begin{funcdesc}{self}{}
Return the \class{Process} instance of the calling process.
\end{funcdesc}

\begin{funcdesc}{self_}{}
An alias for the \function{self()} function. You can use this function in class
methods where \var{self} is already defined.
\end{funcdesc}

\begin{funcdesc}{processes}{}
Return a list of all active processes.
\end{funcdesc}

\begin{funcdesc}{send}{proc, message}
Send the \var{message} to the \var{proc} process and return \var{message}. This
is the same as \var{proc}\code{.send(}\var{message}\code{)}. Raises a
\code{'badarg'} \exception{ExitError} if \var{proc} is not a \class{Process}
instance.
\end{funcdesc}

\begin{funcdesc}{exit}{\optional{proc, }reason}
When the \var{proc} argument is not given, this function raises an
\exception{ExitError} with the reason \var{reason}. \var{reason} can be any
value.

When the \var{proc} argument is given, this function sends an \code{'EXIT'}
signal to the process \var{proc}. This is not necessarily the same as sending an
\code{'EXIT'} message to \var{proc}. They are the same if \var{proc} is trapping
exits. If \var{proc} is not trapping exits, however, the \var{proc} process
terminates and propagates the \code{'EXIT'} signal in turn to its linked
processes.

If the \var{reason} is the string \code{'kill'}, for example
\code{exit(proc, 'kill')}, an untrappable \code{'EXIT'} signal is sent to the
process. In other words, the \var{proc} process is unconditionally killed.

Refer to the \function{processFlag()} function for details about trapping exits.
\end{funcdesc}

\begin{funcdesc}{processFlag}{flag, option}
Set the given \var{flag} for the calling process. Returns the old value of the
flag. Raises a \code{'badarg'} \exception{ExitError} if \var{flag} is not a
recognized flag value, or if \var{option} is not a recognized value for
\var{flag}.

Currently, \code{'trap_exit'} is the only recognized flag value. When
\code{'trap_exit'} is set to \constant{True}, \code{'EXIT'} signals arriving to
a process are converted to \code{('EXIT', from, reason)} messages, which can be
received as ordinary messages. If \code{'trap_exit'} is set to \constant{False},
the process exits if it receives an \code{'EXIT'} signal other than
\code{'normal'} and propagates the \code{'EXIT'} signal to its linked processes.
Application processes should normally not trap exits.
\end{funcdesc}



% ----------------------------------------------------------------------------
\subsection{Process Objects}

\begin{classdesc}{Process}{}
Represents a concurrent process. A \class{Process} is never created via its
constructor. The \function{spawn()} and \function{spawnLink()} functions create
all processes.

\begin{methoddesc}{isProcessAlive}{}
Return \constant{True} if the process is alive, i.e., has not terminated.
Otherwise, return \constant{False}.
\end{methoddesc}

\begin{methoddesc}{isAlive}{}
An alias for the \method{isProcessAlive()} method. (The word ``process'' is
redundant in a method name when the method is a member of the \class{Process}
class.)
\end{methoddesc}

\begin{methoddesc}{send}{message}
Send the \var{message} to this process and return \var{message}. A routine
running in a separate process calls this method to place the given
\var{message} into this process's mailbox. A \class{Receiver} that is operating
in this process may then pick up the message.

Sending a message is an asynchronous operation so the \method{send()} call does
not wait for a \class{Receiver} to retrieve the message. Even if this process
has already terminated, the system does not notify the sender. Messages are
always delivered, and always in the same order they were sent.
\end{methoddesc}

\begin{methoddesc}{__or__}{message}
\opindex{|}
An alias for the \method{send()} method. The OR operator, `\pipe', is aliased to
the \method{send()} method so that developers can use a more Erlangy syntax to
send messages. In Erlang, the `!' primitive sends messages. For example:
\begin{verbatim}
>>> proc | ('knock-knock', 'candygram')
\end{verbatim}
\end{methoddesc}

\end{classdesc}



% ----------------------------------------------------------------------------
\subsection{Receiver Objects}
\label{Receiver}

\begin{classdesc}{Receiver}{}
Retrieves messages from a process's mailbox. Every process maintains a mailbox,
which contains messages that have been sent to it via the \method{send()}
method. To retrieve the messages out of the mailbox, a process must use a
\class{Receiver} object. Like the \keyword{receive} statement in Erlang, a
\class{Receiver} object compares an incoming message against multiple patterns
[\ref{patterns}] and invokes the callback function that is associated
with the first matching pattern.

\begin{notice}[warning]
You may not use a single \class{Receiver} instance across multiple processes.
The \class{Receiver} raises an \exception{AssertionError} if you call one
of its methods from a process other than the one that created it. You should
take care, therefore, to create a \class{Receiver} within the process that will
be using it.
\end{notice}

\begin{methoddesc}{addHandler}{pattern\optional{, func\optional{, args\moreargs}}}
Register a handler function \var{func} for the \var{pattern}. The
\method{receive()} method calls the \var{func} with the \var{args} argument
list when it receives a message that matches \var{pattern}. When a handler
function \var{func} is not specified, the \method{receive()} method removes any
matching message from the mailbox and does nothing more with it. Refer to
section \ref{patterns} for details about how to specify message patterns. Raises
a \code{'badarg'} \exception{ExitError} if \var{func} is not
\function{callable()}.

If any of the \var{args} parameters is \constant{candygram.Message}, then the
\method{receive()} method replaces that parameter with the matching message when
it invokes \var{func}.

Refer to the \method{receive()} documentation for details about the
mechanism by which it invokes the handlers.
\end{methoddesc}

\begin{methoddesc}{__setitem__}{pattern, funcWithArgs}
\opindex{[]}
An alias for the \method{addHandler()} method. If \var{funcWithArgs} is a tuple,
then this method sends the first element in the tuple as the \var{func}
parameter to \method{addHandler()} and the remaining as the \var{args}. If
\var{funcWithArgs} is not a tuple, then this method assumes it to be a handler
function and sends it as the \var{func} parameter to \method{addHandler()}.

\method{__setitem__()} is an alias to the \method{addHandler()} method so that
developers can use a more Erlangy syntax to specify handlers. In Erlang, the
``\var{pattern} -\greaterthan\ \var{func}'' syntax specifies pattern guards. For
example:
\begin{verbatim}
>>> r = Receiver()
>>> r['knock-knock', 'candygram'] = answer, 'From whom?'
\end{verbatim}
\end{methoddesc}

\begin{methoddesc}{__getitem__}{pattern}
\opindex{[]}
An alias for the \method{addHandler()} method. This method calls
\method{addHandler()} without a \var{func} parameter.

\method{__getitem__()} is an alias to the \method{addHandler()} method so that
developers can use a more Erlangy syntax to specify handlers. In Erlang, the
``\var{pattern} -\greaterthan\ \var{func}'' syntax specifies pattern guards. For
example:
\begin{verbatim}
>>> r = Receiver()
>>> r['knock-knock', 'dolphin']  # ignore dolphins
\end{verbatim}
\end{methoddesc}

\begin{methoddesc}{addHandlers}{receiver}
Register all handler functions in \var{receiver} with this \class{Receiver}
object. This method adds all of the patterns and handler functions that have
been added to the given \var{receiver} to this receiver, in the same order. You
can use this method to make copies of \class{Receiver} objects. You may use a
\var{receiver} that was created in a different process. Raises a \code{'badarg'}
\exception{ExitError} if \var{receiver} is not a \class{Receiver} instance.
\end{methoddesc}

\begin{methoddesc}{after}{timeout\optional{, func\optional{, args\moreargs}}}
Register a timeout handler function \var{func}. The \method{receive()} method
calls the \var{func} with the \var{args} argument list when it does not receive
a matching message within \var{timeout} milliseconds. Raises a \code{'badarg'}
\exception{ExitError} if \var{func} is not \function{callable()} or if
\var{timeout} is not an integer. Raises an \exception{AssertionError} if the
\method{after()} method has already been invoked.

If the \method{receive()} method does not find a matching message within
\var{timeout} milliseconds, it invokes \var{func} with \var{args} and returns
the result. When a handler function \var{func} is not specified, the
\method{receive()} method returns \constant{None} if it times out.

None of the \var{args} parameters should be \constant{candygram.Message}, since
there is no message to pass when a timeout occurs.
\end{methoddesc}

\begin{methoddesc}{receive}{\optional{timeout\optional{, func\optional{, args\moreargs}}}}
Find a matching message in the process's mailbox, invoke the related handler
function, and return the result. The \class{Receiver} object compares the first
message in the process's mailbox with each of its registered patterns, in the
order that the patterns were added via the \method{addHandler()} method. If any
pattern matches the message, this method removes the message from the mailbox,
calls the associated handler function, and returns its result. If the matching
pattern does not have a handler function associated with it, this method returns
\constant{None}. If no pattern matches the message, this method will leave the
message in the mailbox, skips to the next message in the mailbox, and compares
it with each of the patterns. It continues on through each of the messages in
the mailbox until a match is found.

If no message matches any of the patterns, or if the mailbox is empty, this
method blocks until the process receives a message that does match one of the
patterns, or until the given \var{timeout} has elapsed. If no \var{timeout} is
specified, this method blocks indefinitely. If a \var{timeout} is specified, it
is equivalent to calling \method{after(\var{timeout}, \var{func}, \var{args})}
just prior to \method{receive()}. This method raises a \code{'badarg'}
\exception{ExitError}, therefore, if \var{func} is not \function{callable()} or
if \var{timeout} is not an integer. It also raises an \exception{AssertionError}
if a \var{timeout} is specified and the \method{after()} method has already been
invoked.
\end{methoddesc}

\begin{methoddesc}{__call__}{\optional{timeout\optional{, func\optional{, args\moreargs}}}}
\opindex{()}
An alias for the \method{receive()} method. \method{__call__()} is an alias to
the \method{receive()} method so that developers can use a shortened, more
Erlangy syntax with \class{Receiver}s. For example:
\begin{verbatim}
>>> def convert():
...     r = Receiver()
...     r['one'] = lambda: 1
...     r['two'] = lambda: 2
...     r['three'] = lambda: 3
...     return r()
\end{verbatim}
\end{methoddesc}

\begin{methoddesc}{__iter__}{}
Return an iterator that repeatedly invokes \method{receive()}. Candygram code
often uses the following idiom:
\begin{verbatim}
... while True:
...     result = receiver.receive()
...     # do whatever with the result...
\end{verbatim}
With an iterator, you can spell the code above like this instead:
\begin{verbatim}
... for result in receiver:
...     # do whatever with the result...
\end{verbatim}
\end{methoddesc}

\end{classdesc}



% ----------------------------------------------------------------------------
\subsection{Constants}

\begin{datadesc}{Any}
When used in a pattern, the \constant{Any} constant will match any value.
\end{datadesc}

\begin{datadesc}{AnyRemaining}
When used in the last position of a tuple or list pattern, the
\constant{AnyRemaining} constant will cause the sequence to match a sequence of
the same kind (tuple or list) that has a length \greaterthan= len(pattern)-1.
The patterns in the sequence prior to \constant{AnyRemaining} must still match
their respective values in a message for the whole sequence to match.
\end{datadesc}

\begin{datadesc}{Message}
When used as a function parameter in \method{Receiver.addHandler()}, the
\method{Receiver.receive()} method replaces the parameter with the matching
message when it invokes the handler function.
\end{datadesc}



% ----------------------------------------------------------------------------
\subsection{Exceptions}

\begin{excdesc}{ExitError}
Represents \code{'EXIT'} errors from Erlang. If an Erlang function can
cause a failure under certain circumstances, then the corresponding Candygram
function raises an \exception{ExitError} under the same circumstances.

A process also raises an \exception{ExitError} in all of its linked processes
if it terminates for a reason other than \code{'normal'}.

\begin{notice}[note]
When a process terminates with a reason other than \code{'normal'}, it does not
immediately raise an \exception{ExitError} in all linked processes. Candygram
defers the \exception{ExitError} instead, and raises it the next time one of its
functions or methods is called. (Python does not allow you to unconditionally
interrupt a separate thread.)
\end{notice}

\begin{memberdesc}{reason}
Specifies the reason why the \exception{ExitError} was raised. Can be any value.
\end{memberdesc}

\begin{memberdesc}{proc}
The process that originally caused the \exception{ExitError}.
\end{memberdesc}

\end{excdesc}



% ############################################################################
\section{Examples}
There is a directory named \file{examples} within every distribution of
Candygram. This directory contains the Candygram equivalents of all the sample
programs found in chapter 5 and sections 7.2, 7.3, and 7.5 of \erlangbook. The
files named \file{program_X.X.py} are direct translations of the Erlang
programs. The files named \file{program_X.X_alt.py} are alternate, more liberal
translations that use more Pythonic idioms. \note{Since the CPython interpreter
does not perform tail-call optimization, the tail-recursive style of the direct
translations is not a recommended practice.}

The \file{examples} directory also contains a few other modules that demonstrate
how you can Candygram to perform some handy functions.

If you are not already familiar with Erlang, the best way to become familiar
with Candygram is to read chapter 5 in \erlangbook\ and follow along using the
example programs located in the \file{examples} directory. If you are already
familiar with Erlang, here are a few code snippets to give you a taste of
Candygram.

\begin{verbatim}
>>> from candygram import *
>>> def proc_func():
...     r = Receiver()
...     r['land shark'] = lambda m: 'Go Away ' + m, Message
...     r['candygram'] = lambda m: 'Hello ' + m, Message
...     for message in r:
...         print message
...
>>> proc = spawn(proc_func)
>>> proc | 'land shark'
>>> proc | 'candygram'
\end{verbatim}
Running the code above produces the following output:
\begin{verbatim}
Go Away land shark
Hello candygram
\end{verbatim}

The code above uses a rather Erlangy syntax. Here is a more Pythonic version
that does the same:
\begin{verbatim}
>>> import candygram as cg
>>> def proc_func():
...     r = cg.Receiver()
...     r.addHandler('land shark', shut_door, cg.Message)
...     r.addHandler('candygram', open_door, cg.Message)
...     for message in r:
...         print message
...
>>> def shut_door(name):
...     return 'Go Away ' + name
...
>>> def open_door(name):
...     return 'Hello ' + name
...
>>> proc = cg.spawn(proc_func)
>>> proc.send('land shark')
>>> proc.send('candygram')
\end{verbatim}

Lastly, here is an example with more elaborate patterns:
\begin{verbatim}
>>> from candygram import *
>>> def proc_func(name):
...     r = Receiver()
...     r['msg', Process, str] = print_string, name, Message
...     r['msg', Process, str, Any] = print_any, name, Message
...     r[Any]  # Ignore any other messages
...     for result in r:
...         pass
...
>>> def print_string(name, message):
...     msg, process, string = message
...     # 'msg' and 'process' are unused
...     print '%s received: %s' % (name, string)
...
>>> def print_any(name, message):
...     msg, process, prefix, value = message
...     # 'msg' and 'process' are unused
...     print '%s received: %s %s' % (name, prefix, value)
...
>>> a = spawn(proc_func, 'A')
>>> b = spawn(proc_func, 'B')
>>> a | ('msg', b, 'Girl Scout cookies')
>>> a | 'plumber?'
>>> a | ('msg', b, 'The meaning of life is:', 42)
\end{verbatim}
Running the code above produces the following output:
\begin{verbatim}
A received: Girl Scout cookies
A received: The meaning of life is: 42
\end{verbatim}



% ############################################################################
\section{FAQ}

\subsection{Why is the package called Candygram?}
The name Candygram is actually an acronym for ``the Candygram Acronym Does Not
Yield a Good Reference to Anything Meaningful.''

\subsection{But wait, doesn't that spell {\sc cadnygram}?}
Yes, you are quite observant. In order to form a compromise with the French
acronym, which is {\sc canydgram}, the Candygram committee standardized the
official acronym as {\sc Candygram}.

\subsection{How do you pronounce Candygram?}
This question produces an outrageous amount of heated debate. Some claim that
you pronounce it with short A's, as in tomato, while others claim that you
pronounce it with long A's, as in potato. Both sides, however, are completely
wrong; the correct pronunciation for Candygram is ``throat warbler mangrove.''



% ############################################################################
\section{Feedback}

Please submit all bug reports, feature requests, etc. to the appropriate tracker
on \ulink{SourceForge.net}{http://sourceforge.net/tracker/?group_id=114295}.

General discussion takes place on the
\email{candygram-discuss@lists.sourceforge.net} mailing list. You can subscribe
to this list by visiting \ulink{this page}
	{http://lists.sourceforge.net/lists/listinfo/candygram-discuss}.



% ############################################################################
\appendix
\section{GNU Free Documentation License}
\label{fdl}
\input{LICENSE.tex}



% ############################################################################
\input{candygram.ind}
\end{document}

\end{document}

\end{document}

\end{document}
